% Seta a imagem do capítulo
\chapterimage{chapter_head_2.pdf}
% O título e rótulo do cabiluto
\chapter{Lógica Proposicional}\label{cap:LogicaProposicional}

\epigraph{``Ou a matemática é muito grande para a mente humana, ou a mente humana é mais do que uma máquina.''}{Kurt Gödel}

\section{A linguagem proposicional}

Este capítulo tem como objetivo apresentar ao leitor o cálculo proposicional, ou seja, o estudo da lógica proposicional, em seus dois aspectos já bem estabelecido por matemáticos e filósofos, isto é,  sua sintaxe e sua semântica\footnote{O aspecto pragmático da lógica, por ainda se encontrar em um estágio primitivo de seu desenvolvimento, do ponto de vista matemático, não será abordado neste texto, para este assunto ver \cite{rodrigues2021, silva2018}.}. Assim este capítulo começa com a formalização da linguagem da lógica proposicional, isto é, a linguagem proposicional. A seguir é apresentado formalmente a noção de alfabeto proposicional.

\begin{definition}[Alfabeto Proposicional]\label{def:AlfProp}
    O alfabeto proposicional corresponde ao conjunto enumerável $\Sigma = \Sigma_s \cup \Sigma_o \cup \Sigma_p \cup \{\bot\}$ onde:
    \begin{itemize}
        \item $\Sigma_s = \{A, \cdots, P, Q, R, P_1, Q_{12}, \cdots\}$ é um conjunto enumerável, chamado conjunto de símbolos proposicionais;
        \item $\Sigma_o = \{\land, \lor, \neg, \Rightarrow\}$ é o conjunto dos símbolos operacionais\footnote{Também é comum encontrar na literatura (ver \cite{joaoPavao2014}) a nomenclatura conjunto de conectivos.};
        \item $\Sigma_p = \{(, )\}$ é o conjunto dos símbolos de pontuação e
        \item $\bot$ é o símbolo do absurdo.
    \end{itemize}
\end{definition}

Qualquer sequência de símbolos do alfabeto proposicional é chamada de palavra, entretanto, nem toda palavra será considerada como sendo parte da linguagem proposicional. A Definição \ref{def:LingProp} a seguir formaliza o conjunto que corresponde a linguagem proposicional.

\begin{definition}[Linguagem Proposicional]\label{def:LingProp}
    Dado o alfabeto proposicional $\Sigma$ a linguagem proposicional, denotada por $\mathcal{L}_{prop}$, é o menor conjunto indutivamente gerado pelas seguintes regras:
    \begin{enumerate}
        \item Para todo $\alpha \in \Sigma_s \cup \{\bot\}$, tem-se que $\alpha \in \mathcal{L}_{prop}$.
        \item Se $\alpha \in \mathcal{L}_{prop}$, então $(\neg\alpha) \in \mathcal{L}_{prop}$.
        \item Se $\alpha, \beta \in \mathcal{L}_{prop}$, então $(\alpha \land \beta), (\alpha \lor \beta), (\alpha \Rightarrow \beta) \in \mathcal{L}_{prop}$. 
    \end{enumerate}
\end{definition}

\begin{exem}\label{exe:PalavrasProposicionaisBemFormadas}
    Dado $P, Q, R, S, T \in \Sigma_s \cup \{\bot\}$ tem-se que:
    \begin{itemize}
        \item[(a)] $P$
        \item[(b)] $(P \land Q)$
        \item[(c)] $(R \Rightarrow S)$
        \item[(d)] $((Q \lor S) \Rightarrow T)$
    \end{itemize}
    são todas palavras da linguagem $\mathcal{L}_{prop}$. Por outro lado, as palavras:
    \begin{itemize}
        \item[(e)] $P \land$
        \item[(f)] $\Rightarrow Q$
        \item[(g)] $P \lor \land Q$
    \end{itemize}
    não são palavras da linguagem $\mathcal{L}_{prop}$.
\end{exem}

\begin{rema}
    Para simplificar a escrita das palavras de $\mathcal{L}_{prop}$ é comum omitirem-se os parênteses mais exteriores das palavras, por exemplo, é escrito apenas $(Q \lor S) \Rightarrow T$ em vez de $((Q \lor S) \Rightarrow T)$. 
\end{rema}

É possível enriquecer\footnote{No sentido de adicionar mais símbolos operacionais.} a linguagem proposicional adicionando mais símbolos operacionais no alfabeto da mesma, essa introdução é feita utilizando o conceito de abreviação. Uma abreviação na lógica formal consiste na ação de usar um novo símbolo para criar uma nova palavra não presente originalmente na linguagem proposicional, mas que representa uma palavra da linguagem. 

\begin{definition}[Abreviatura da Bi-implicação]\label{def:SeSomenteSeAbreviatura}
    Considerando o novo símbolo $\Leftrightarrow$, para todo $\alpha, \beta \in \mathcal{L}_{prop}$ as palavras na forma $\alpha \Leftrightarrow \beta$ são as abreviações das palavras de $\mathcal{L}_{prop}$ na forma $(\alpha \Rightarrow \beta) \land (\beta \Rightarrow \alpha)$, em notação formal tem-se $\alpha \Leftrightarrow \beta \equiv_{abr} (\alpha \Rightarrow \beta) \land (\beta \Rightarrow \alpha)$.
\end{definition}

De fato, muitos dos símbolos operacionais que foram tomados como símbolos básicos do alfabeto proposicional (Definição \ref{def:AlfProp}) poderiam ser removidos, pois como muito bem explicado em \cite{BenjaV1, joaoPavao2014} a lógica proposicional pode ser definida sobre a linguagem que contém apenas os símbolos operacionais de $\Rightarrow$ e $\neg$, os demais símbolos podem ser obtidos via abreviação sem qualquer perda no estudo da lógica proposicional, para mais detalhes ver \cite{BenjaV1}.

\section{Sistema Dedutivo}\label{sec:SistemaDedutivo}

A ideia de sistemas dedutivos para a lógica formal remonta aos trabalhos publicados\footnote{Esses trabalhos podem ser encontrados re-editados respectivamente em \cite{gentzen1969} e \cite{jaskowski1934}.} no ano de 1934 pelo matemático e filósofo alemão Gerhard Gentzen (1909-1945) e pelo lógico polonês Stanisław Jaśkowski (1906-1965). Existem diversos sistemas dedutivos para a lógica proposicional, cada um possuindo suas próprias características, vantagens e desvantagens, no entanto, todos os sistemas dedutivos compartilham a característica em comum de possuírem um conjunto finito de regras de inferência, esse conjunto de regras de inferência é também chamado de sistema regras ou sistema de dedução \cite{edgar2002}.

O sistema dedutivo introduzido por Gentzen e Jaśkowski é conhecido por dedução natural, aqui ele será apresentado de forma similar a exposição feita em \cite{joaoPavao2014}. O conjunto de regras de inferência da dedução natural e composto pelas regras: de introdução e eliminação de conectivos, regra de reiteração, introdução de hipóteses e a regra do absurdo. Entretanto, antes de apresentar as regras do sistema de dedução natural e conveniente apresentar o conceito de demonstração, para isso deve-se escolher uma notação para as provas da dedução natural.

Existem diversas formas de se escrever (ou representar) uma demonstração no sistema de dedução natural, entre elas destacam-se as árvores de prova de Gentzen \cite{BenjaV1}, o estilo linear \cite{copi1981, mortari2001} e o estilo de Fitch \cite{joaoPavao2014, fitch1953}. 

Neste texto será adotado o estilo de Fitch como modelo padrão para a escrita das demonstrações do sistema de dedução natural para a lógica proposicional, assim é conveniente apresentar de forma sucinta o estilo de Fitch.

O estilo de Fitch foi introduzido pelo lógico americano Frederic Brenton Fitch (1908 - 1987) e corresponde a diagramas hierárquicos formados por linhas e barras (verticais e horizontais) que representam o raciocínio para a partir de um conjunto de premissas se obter uma determinada conclusão ou objetivo (em inglês \textit{goal}).

O diagrama de Fitch é organizado por linhas numeradas, onde cada linha $i$ pode conter uma única palavra de $\mathcal{L}_{prop}$, sendo essa palavra uma premissa ou sendo ela obtida pela aplicação de alguma regra de inferência sobre uma ou mais linhas anteriores a linha $i$. 

As barras verticais nos diagramas de Fitch são usadas de duas formas:
\begin{itemize}
    \item[(1)] Para separar a demonstração em escopos, sendo que um escopo consiste de uma sequencia de várias linhas (ou passos) para demonstrar uma conclusão.
    \item[(2)] Como um mecanismo para saber quais palavras de $\mathcal{L}_{prop}$ estão ativas\footnote{Uma palavra de $\mathcal{L}_{prop}$ está ativa em uma demonstração, enquanto o escopo da mesma está aberto na demonstração.} na prova, como explicado em \cite{joaoPavao2014}. 
\end{itemize}

As barras horizontais no diagrama de Fitch indicam a divisão entre  as  afirmações  que  estamos  assumindo  (nossas premissas e (ou) hipóteses) e as palavras que se seguem delas, sejam conclusões intermediárias ou nosso objetivo final. No caso das hipóteses a barra horizontal também cria um novo ``escopo'', isto é, adiciona uma indentação em relação ao escopo anterior, vale salientar que cada escopo é na verdade uma prova para um (sub-)objetivo. 

Por fim, é comum na notação dos diagramas de Fitch escrever mais à direita de cada linha a regra de inferência que gerou a palavra na linha, ou o fato da palavra ser uma premissa ou hipótese. Agora pode-se apresentar formalmente o conceito de prova que será adotado neste capítulo.

\begin{definition}[Prova]\label{def:Prova}
    Uma prova para $\alpha \in \mathcal{L}_{prop}$ consiste de um diagrama de Fitch como uma quantidade finita de linhas, de forma que a última linha contém a palavra $\alpha$ e cada linha $i$ anterior contém uma palavra $\beta_i \in \mathcal{L}_{prop}$ tal que $\beta_i$ ou é uma premissa ou é obtida via aplicação de alguma regra de inferência.
\end{definition}

Agora pode-se definir precisamente o conceito de relação de consequência sintática sobre a linguagem $\mathcal{L}_{prop}$.

\begin{definition}[Consequência Sintática]\label{def:ConseSintatica}
    Seja $\mathcal{L}_{prop}$ a linguagem proposicional, dado $\alpha \in \mathcal{L}_{prop}$ e $\Gamma \subseteq \mathcal{L}_{prop}$, diz-se que $\alpha$ é consequência sintática de $\Gamma$, denotado por $\Gamma \vdash \alpha$, sempre que existir uma prova de $\alpha$ a partir do conjunto de premissas $\Gamma$. 
\end{definition}

\begin{rema}
    Note que uma instância de consequência sintática pode ser vista como um elemento de $\wp(\mathcal{L}_{prop}) \times \mathcal{L}_{prop}$, isto é, a  consequência sintática $(\vdash)$ pode ser vista como uma relação no sentido usual da teoria ingênua dos conjuntos.
\end{rema}

A seguir são apresentadas as regras de inferência do sistema de dedução natural, aqui será iniciada pelas regras que não envolvem diretamente os símbolo operacionais, isto é, que não age diretamente para eliminar ou introduzir os elementos de $\Sigma_o$ na demonstração.

\begin{definition}[Regra das premissas]\label{def:RegraPremissas}
    Se $\Gamma = \{\alpha_1, \cdots, \alpha_n \}$ é um conjunto finitos de premissas, então a regra das premissas fixa que a construção do diagrama de Fitch para uma prova de $\Gamma \vdash \alpha$ dispões nas $n$ primeiras linhas do diagrama as $n$ premissas contidas $\Gamma$, onde na linha $i$ se encontra a premissa $\alpha_i$, além disso, existe uma barra vertical contínua\footnote{Cada linha vertical contínua é um escopo dentro da demonstração.} a esquerda das premissas e após a linha $n$ há uma barra horizontal separando as promissas do resto da prova, ou seja:
    $$
    \begin{nd}
        \have[1]{h}{\alpha_1} \by{Premissa}{}
        \have[\vdots]{skip1}{\vdots} 
        \hypo[n]{atob}{\alpha_n} \by{Premissa}{}
        \have[\vdots]{skip1}{\vdots}
    \end{nd}
    $$
\end{definition}

\begin{exem}\label{exe:RegraPremissas}
    A prova de $\{P, Q\} \vdash P \land Q$ pode ser iniciada usando a regra das premissas de forma que é obtido o seguinte diagrama inicial:
    $$
        \begin{nd}
            \have[1]{h}{P} \by{Premissa}{}
            \hypo[2]{atob}{Q} \by{Premissa}{}
        \end{nd}
   $$
\end{exem}

Seguindo com as regras mais básicas do sistema de dedução natural tem-se a regra de reiteração, repetição, copia ou clonagem, aqui esta regra será denotada apenas por REI.

\begin{definition}[Regra da reiteração]\label{def:RegraRepetição}
    Em uma demonstração sempre é possível repetir uma palavra $\beta \in \mathcal{L}_{prop}$ que já foi obtida em uma linha $i$ durante a prova, desde que o escopo que contém $\beta$ ainda esteja ativo\footnote{A noção de escopo ativo diz respeito se uma (sub-)prova foi concluída ou ainda está em desenvolvimento, este conceito será melhor trabalhado mais adiante.}. Na notação de Fitch tem-se:
    $$
        \begin{nd}
            \have[\vdots]{skip1}{\vdots} 
            \have[i]{h}{\beta}
            \have[\vdots]{skip1}{\vdots} 
            \have[n]{atob}{\beta} \by{REI}{h}
            \have[\vdots]{skip1}{\vdots}
        \end{nd}
    $$
\end{definition}

\begin{exem}\label{exe:AplicacaoCopia}
    Em uma prova de $\{P, Q\} \vdash P \land(P \land Q)$ após aplicar a regra das premissas pode-se aplicar a regra de reiteração na linha 1 e com isso é obtido uma segunda ``instância'' da proposição $P$:
    $$
        \begin{nd}
            \have[1]{h}{P} \by{Premissa}{}
            \hypo[2]{atob}{Q} \by{Premissa}{}
            \have[3]{b}{P} \by{REI}{h}
        \end{nd}
   $$
\end{exem}

Agora que já foram apresentadas as regras que não agem diretamente sobre os símbolos operacionais pode-se dá sequência no texto apresentando as regras de inferência do sistema de dedução natural que atuam diretamente sobre os símbolos. 

\begin{definition}[Regra de introdução da conjunção ($\land I$)]\label{def:RegraIntroducaoE}
    Se em uma prova foram deduzidas as palavras $\alpha, \beta \in \mathcal{L}_{prop}$ nas linhas $i$ e $j$ respectivamente, então pode-se deduzir a palavra $\alpha \land \beta$ em uma linha $k$ com $i < j < k$, na notação do diagrama de Fitch tem-se:
    $$
        \begin{nd}
            \have[\vdots]{skip1}{\vdots} 
            \have[i]{a}{\alpha}
            \have[\vdots]{skip1}{\vdots} 
            \have[j]{b}{\beta} 
            \have[\vdots]{skip1}{\vdots} 
            \have[k]{c}{\alpha \land \beta} \ai{a, b}
            \have[\vdots]{skip1}{\vdots}
        \end{nd}
    $$
\end{definition}

\begin{rema}
    A regra de introdução da conjunção impõe que a palavra que está na linha $i$ seja fixada à esquerda do símbolo $\land$ e a palavra na linha $j$ seja fixada à direita do símbolo $\land$.
\end{rema}

\begin{exem}\label{exe:RegraIntroducaoE}
    Para concluir a prova de $\{P, Q\} \vdash P \land Q$ iniciada no Exemplo \ref{exe:RegraPremissas} basta aplicar a regra de introdução da conjunção nas linhas 1 e 2, como pode ser visto a seguir.
    $$
        \begin{nd}
            \have[1]{a}{P} \by{Premissa}{}
            \hypo[2]{b}{Q} \by{Premissa}{}
            \have[3]{c}{P \land Q} \ai{a, b}
        \end{nd}
    $$
\end{exem}

A próxima regra é a eliminação da conjunção, tal regra possui duas formas o que contrasta com a regra da introdução da conjunção que possui apenas uma única forma, note que o operador $\land$ combina duas palavras $\alpha, \beta \in \mathcal{L}_{prop}$, assim quando tal operador for removido deve-se optar por qual das duas palavras será mantida como uma conclusão (intermediária ou final) da prova. A seguir é definida formalmente a regra de eliminação de conjunção.

\begin{definition}[Regra de eliminação da conjunção ($\land E$)]\label{def:RegraEliminacaoE}
    Se em uma prova for deduzida a palavra $\alpha \land \beta$ na linha $i$, então pode-se deduzir a palavra $\alpha$ ou então a palavra $\beta$ em uma linha $j$ com $i < j$, na notação do diagrama de Fitch tem-se:
    
    \begin{minipage}{.45\textwidth} %
        $$
            \begin{nd}
                \have[\vdots]{skip1}{\vdots}  
                \have[i]{a}{\alpha \land \beta}
                \have[\vdots]{skip1}{\vdots}  
                \have[j]{b}{\alpha} \ae{a}
                \have[\vdots]{skip1}{\vdots} 
            \end{nd}
        $$
    \end{minipage} %
    ou
    \begin{minipage}{.45\textwidth} %
        $$
            \begin{nd}
                \have[\vdots]{skip1}{\vdots}  
                \have[i]{a}{\alpha \land \beta}
                \have[\vdots]{skip1}{\vdots}  
                \have[j]{b}{\beta} \ae{a}
                \have[\vdots]{skip1}{\vdots} 
            \end{nd}
        $$
    \end{minipage}
\end{definition}

\begin{exem}\label{exe:RegraEliminacaoE}
    Uma prova de $\{(P \land Q) \land R\} \vdash P$ é dada pelas aplicação da eliminação da conjunção duas vezes seguidas como pode ser visto a seguir.
    $$
        \begin{nd}
            \hypo[1]{a}{(P \land Q) \land R} \by{Premissa}{}
            \have[2]{b}{P \land Q} \ae{a}
            \have[3]{c}{P} \ae{b}
        \end{nd}
    $$
\end{exem}

\begin{rema}
    No Exemplo \ref{exe:RegraEliminacaoE} na linha 2 foi escrito apenas $P \land Q$ em vez de $(P \land Q)$, isto é permitido pois como dito anteriormente para simplificar a escrita sempre que não causar confusão os parêntese mais externos podem ser removidos.
\end{rema}

Agora será aberto um parêntese na apresentação das regras de inferência dos símbolos operacionais para que possa ser discutido neste texto a noção de prova hipotética. As provas hipotéticas são muito importantes dentro do sistema de dedução natural, tais provas com dito em \cite{joaoPavao2014}, podem ser pensadas como sendo um ambiente (ou escopo) de sub-prova em que além das premissas que iniciaram a prova são assumidas outras informações na forma de hipóteses. 

Como argumentado em \cite{copi1981, joaoPavao2014}, uma prova hipotética surge quando a regra de introdução hipótese é aplicada, e ao se introduzir essa nova hipótese na prova é gerado um novo escopo dentro da prova que se estava demonstrando, isto é, é criada uma sub-prova que terá seu próprio objetivo. 

\begin{definition}[Regra de introdução de hipótese]\label{def:RegraHipotese}
    Dado uma demonstração com $n$ passos, se for necessário assumir uma hipótese $\beta \in \mathcal{L}_{prop}$ no passo $n+1$, então é inserida a hipótese $\beta$ junto com uma barra vertical de escopo, e abaixo de $\beta$ é inserida a barra horizontal de separação para destacar a hipótese, aqui será usado a palavra \textbf{Assuma} para referenciar a regra de introdução de hipótese\footnote{Na literatura em língua inglesa é comum o uso do termo \textit{Assumption}.}.
    $$
        \begin{nd}
            \have[\vdots]{skip1}{\vdots}  
            \have[n]{a}{\vdots}
            \open
            \hypo[n+1]{b}{\beta} \by{Assuma}{}  
            \have[\vdots]{c}{\vdots}
            \close
        \end{nd}
    $$
\end{definition}

Como dito em \cite{edgar2002}, uso da regra de inferência de introdução de hipótese está intimamente ligada ao uso da regra de introdução da implicação definida a seguir, por isso a necessidade de apresenta-lá antes da regra de introdução da implicação. 

\begin{definition}[Introdução da implicação ($\Rightarrow I$)]\label{def:RegraIntroImplicacao}
    Se partindo de uma suposição hipotética $\alpha$ na linha $m$ for possível deduzir um certo $\beta$ na linha $n$ com $m < n$, então no escopo externo da prova hipotética é concluído na linha $n+1$ que $\alpha \Rightarrow \beta$, na notação dos diagrama de Fitch tem-se:
    
    $$
        \begin{nd}
            \have[\vdots]{skip1}{\vdots}  
            \open
            \hypo[m]{a}{\alpha} \by{Assuma}{}  
            \have[\vdots]{b}{\vdots}
            \have[n]{c}{\beta}
            \close
            \have[n+1]{d}{\alpha \Rightarrow \beta} \ii{a--c}
            \have[\vdots]{skip1}{\vdots} 
        \end{nd}
    $$
\end{definition}

\begin{rema}
    Note que a regra de introdução da implicação pode ser vista como um mecanismo que desativa um escopo de prova, isto é, quando a mesma é aplicada um escopo de prova terá sido completado e assim estará desativado.
\end{rema}

\begin{exem}\label{exe:RegraIntroducaoImplicacao}
    Para provar que a $P \Rightarrow P$ é consequência sintática de um conjunto vazio de premissas utiliza-se a combinação das regras de introdução de hipótese, reiteração e da introdução da implicação como pode ser visto pelo diagrama a seguir.
    $$
        \begin{nd}
            \hypo{a}{P} \by{Assuma}{}  
            \have{b}{P} \by{REI}{a}
            \close
            \have{c}{P \Rightarrow P} \ii{a--b}
        \end{nd}
    $$
\end{exem}

\begin{rema}
    Ao desativar um escopo de prova todas as palavras contidas entre as linhas $i$ e $j$, que forma a prova, não podem mais ser utilizadas na sequência da demonstração, isso ocorre pela razão de tais palavras só existirem no escopo ``local'' da sub-prova que foi concluída.
\end{rema}

Aproveitando o Exemplo \ref{exe:RegraIntroducaoImplicacao}, antes de seguir o texto com a próxima regra de inferência é interessante introduzir ao leitor a ideia de teorema, este conceito é extremamente importante no estudo de qualquer lógica e o mesmo é descrito formalmente a seguir.

\begin{definition}[Teorema]
    Seja $\mathcal{L}$ uma linguagem formal\footnote{A palavra formal aqui diz respeito a ideia de sabe-se precisamente a forma de todas as palavras contidas na linguagem.} e seja $\vdash$ uma relação de consequência sintática sobre $\mathcal{L}$, uma palavra $\alpha \in \mathcal{L}$ é dita ser um teorema sempre que $\emptyset \vdash \alpha$\footnote{É também comum encontrar na literatura a notação $\vdash \alpha$ em vez de $\emptyset \vdash \alpha$.}.
\end{definition}

\begin{rema}
    Dizer que $\alpha$ é um teorema, significa que $\alpha$ é uma consequência direta do próprio sistema sintático da linguagem, isto é, que $\alpha$ é consequência das próprias regras de inferência, sem que haja a necessidade da existência de premissas.  
\end{rema}

Usando apenas as regras de inferência apresentadas até este ponto do texto no próximo exemplo será mostrado um clássico teorema da linguagem proposicional.

\begin{exem}
    Para qualquer $\alpha, \beta \in \mathcal{L}_{prop}$ tem-se o seguinte diagrama de Fitch:
     $$
        \begin{nd}
            \hypo{a}{\alpha \land \beta } \by{Assuma}{}  
            \have{b}{\beta} \ae{a}
            \have{c}{\alpha} \ae{a}
            \have{d}{\beta \land \alpha} \ai{b,c}
            \close
            \have{f}{(\alpha \land \beta) \Rightarrow (\beta \land \alpha)} \ii{a--d}
        \end{nd}
    $$
    Portanto, para qualquer $\alpha, \beta \in \mathcal{L}_{prop}$ tem-se que $ \vdash \alpha \land \beta \Rightarrow \beta \land \alpha$, ou seja, a palavra $(\alpha \land \beta) \Rightarrow (\beta \land \alpha)$ é um teorema da linguagem $\mathcal{L}_{prop}$.
\end{exem}

Prosseguindo com a apresentação das regras de inferência do sistema de dedução natural a seguir será definida formalmente a regra de eliminação da implicação, também conhecida como \textit{modus ponens}, que surge da expressão em latin, \textit{modus ponendo ponens}, que em português pode ser traduzido como: \textbf{o modo de afirmar, afirmando}. 

\begin{definition}[Regra da eliminação da implicação ($\Rightarrow$E)]\label{def:EliminacaoImplicacao}
    Se em uma prova na linha $i$ existe uma palavra $\alpha$ e em uma linha $j$ existe uma palavra $\alpha \Rightarrow \beta$ com $i < j$, então na linha $k$ tal que $j < k$ é possível deduzir a palavra $\beta$, em diagrama tem-se:
    $$
        \begin{nd}
            \have[i]{a}{\alpha}
            \have[\vdots]{skip1}{\vdots}  
            \have[j]{b}{\alpha \Rightarrow \beta}
            \have[\vdots]{skip1}{\vdots} 
            \have[k]{c}{\beta} \ie{a,b}
            \have[\vdots]{skip1}{\vdots}
        \end{nd}
   $$
\end{definition}

\begin{rema}
    O leitor deve ficar atento ao fato de que a Definição \ref{def:EliminacaoImplicacao} especifica que o termo hipotético $\alpha$ deve aparecer na prova antes do termo condicional $\alpha \Rightarrow \beta$, para que se possa aplicar a regra $\Rightarrow$E.
\end{rema}

\begin{exem}\label{exe:RegraEliminacaoImplicacao}
     A prova de $\{P \Rightarrow Q, P \land R\} \vdash Q$ é dado pelo seguinte diagrama:
     $$
        \begin{nd}
            \have{a}{P \Rightarrow Q} \by{Premissa}{}
            \hypo{b}{P \land R} \by{Premissa}{}
            \have{c}{P} \ae{b}
            \have{d}{P \Rightarrow Q} \by{REI}{a}
            \have{e}{Q} \ie{c,d}
        \end{nd}
    $$
\end{exem}

A próxima regra de inferência do sistema de dedução natural que será apresentada neste texto é chamada de regra de introdução do absurdo, a mesma é utilizada para introduzir na demonstração o símbolo do absurdo\footnote{É comum na literatura em língua inglesa principalmente na área de lógica algébrica achar o símbolo do absurdo sendo chamado  \textit{bottom}.} $(\bot)$.

\begin{definition}[Regra de introdução do absurdo ($\bot I$)]\label{def:IntroducaoDoAbsurdo}
     Se na linha  $i$ de uma prova existe uma palavra $\beta$ e no mesmo escopo de prova na linha $j$ existe uma palavra $\neg \beta$ com $i < j$, então na linha $k$ desta prova é deduzido $\bot$ com $j < k$, em diagrama tem-se:
    $$
        \begin{nd}
            \have[\vdots]{skip1}{\vdots} 
            \have[i]{a}{\beta}
            \have[\vdots]{skip1}{\vdots} 
            \have[j]{b}{\neg \beta} 
            \have[\vdots]{skip1}{\vdots} 
            \have[k]{c}{\bot} \by{$\bot$ I}{a,b}
            \have[\vdots]{skip1}{\vdots} 
        \end{nd}
    $$
\end{definition}

\begin{exem}\label{exe:RegraIntroducaoDoAbsurdo}
     Uma prova de $\{P \land Q, R \land \neg P\} \vdash \bot$ é dado pelo seguinte diagrama:
     $$
        \begin{nd}
            \have{a}{P \land Q} \by{Premissa}{}
            \hypo{b}{R \land \neg P} \by{Premissa}{}
            \have{c}{P} \ae{a}
            \have{d}{\neg P} \ae{b}
            \have{e}{\bot} \by{$\bot$ I}{c,d}
        \end{nd}
    $$
\end{exem}

\begin{rema}
    O leitor deve ficar atendo ao fato da Definição \ref{def:IntroducaoDoAbsurdo} estabelecer que a palavra $\alpha$ deve vim antes da palavra $\neg \alpha$ no escopo da prova.
\end{rema}

Seguindo com o desenvolvimento do sistema de dedução natural neste texto, agora serão apresentadas as regras de inferência responsáveis por introduzir e eliminar o símbolo da negação.

\begin{definition}[Regra de introdução da negação ($\neg I$)]\label{def:IntroducaoNegacao}
    Se existe uma sub-prova iniciada com $\alpha$ na linha $i$ que deduz $\bot$ em uma linha $j$  tal que $i < j$, então na linha $k$ com $j < k$ da prova que contém a sub-prova pode-se introduzir a palavra $\neg \alpha$. Em notação de diagrama de Fitch tem-se:
    $$
        \begin{nd}
            \have[\vdots]{skip1}{\vdots}
            \open
            \hypo[i]{a}{\alpha}
            \have[\vdots]{skip1}{\vdots}
            \have[j]{b}{\bot}
            \close
            \have[\vdots]{skip1}{\vdots}
            \have[k]{c}{\neg \alpha} \ni{a--b}
            \have[\vdots]{skip1}{\vdots}
        \end{nd}
    $$
\end{definition}

\begin{exem}\label{exe:IntroducaoNegacao}
     Uma prova de $\{P \Rightarrow \neg Q, Q\} \vdash \neg P$ é descrita pelo diagrama a seguir.
     $$
        \begin{nd}
            \have{a}{P \Rightarrow \neg Q} \by{Premissa}{}
            \hypo{b}{Q} \by{Premissa}{}
            \open
            \hypo{c}{P} \by{Assuma}{}
            \have{d}{P \Rightarrow \neg Q}  \by{REI}{a}
            \have{e}{Q}  \by{REI}{b}
            \have{f}{\neg Q} \ie{c,d}
            \have{g}{\bot} \by{$\bot$ I}{e,f}
            \close
            \have{h}{\neg P} \ni{c--g}
        \end{nd}
    $$
\end{exem}

\begin{definition}[Regra de eliminação da negação ($\neg E$)]\label{def:EliminacaoDaNegacao}
    Sempre que existir uma palavra $\neg \neg \alpha$ em uma linha $i$, então em uma linha $j$ pode-se deduzir $\alpha$ com $i < j$. Em notação de diagrama tem-se:
    $$
        \begin{nd}
            \have[\vdots]{skip1}{\vdots}
            \have[i]{a}{\neg \neg \alpha}
            \have[\vdots]{skip1}{\vdots}
            \have[j]{b}{\alpha} \ne{a}
            \have[\vdots]{skip1}{\vdots}
        \end{nd}
    $$
\end{definition}

E possível interpretar esta regra como representado a ideia de que negar uma palavra (argumento) duas vezes é o mesmo que afirmar tal palavra (argumento).

\begin{exem}
    O seguinte resultado $\vdash (P \land \neg P) \Rightarrow Q$ é um dos mais famosos e controversos teoremas envolvendo o operador de implicação, para detalhes ver \cite{joaoPavao2014}, na demostração pode-se ver o uso da regra de eliminação da negação para provar tal teorema.
    
     $$
        \begin{nd}
            \hypo{a}{P \land \neg P} \by{Assuma}{}
            \open
            \hypo{b}{\neg Q} \by{Assuma}{}
            \have{c}{P \land \neg P}  \by{REI}{a}
            \have{d}{P} \ae{c}
            \have{e}{\neg P} \ae{c}
            \have{g}{\bot} \by{$\bot$ I}{d,e}
            \close
            \have{h}{\neg \neg Q} \ni{b--g}
            \have{i}{Q} \ne{h}
            \close
            \have{j}{(P \land \neg P) \Rightarrow Q} \ii{a--i}
        \end{nd}
    $$
\end{exem}

Por fim, serão agora apresentadas as regras de introdução e eliminação para a disjunção para o sistema de dedução natural.

\begin{definition}[Regra de introdução da disjunção ($\lor I$)]\label{def:IntroducaoDisjuncao}
    Se em uma prova aparece na linha $i$ uma palavra $\alpha$, então em uma linha $j$ tal que $i < j$ pode-se deduzir para algum $\beta \in \mathcal{L}_{prop}$ uma das seguintes palavras: $\alpha \lor \beta$ ou $\beta \lor \alpha$. Na notação do diagrama de Fitch tem-se:
    
    \begin{minipage}{.40\textwidth} %
        $$
            \begin{nd}
                \have[\vdots]{skip1}{\vdots}  
                \have[i]{a}{\alpha}
                \have[\vdots]{skip1}{\vdots}  
                \have[j]{b}{\alpha \lor \beta} \oi{a}
                \have[\vdots]{skip1}{\vdots} 
            \end{nd}
        $$
    \end{minipage} %
    ou
    \begin{minipage}{.40\textwidth} %
        $$
            \begin{nd}
                \have[\vdots]{skip1}{\vdots}  
                \have[i]{a}{\alpha}
                \have[\vdots]{skip1}{\vdots}  
                \have[j]{b}{\beta \lor \alpha} \oi{a}
                \have[\vdots]{skip1}{\vdots} 
            \end{nd}
        $$
    \end{minipage}
\end{definition}

\begin{exem}\label{exe:IntroducaoDisjuncao}
     Uma simples prova da relação de consequência sintática $\{\neg S \Rightarrow (P \land Q), \neg S\} \vdash Q \lor \neg R$ utilizando a regra de introdução da disjunção pode ser vista a seguir.
     $$
        \begin{nd}
            \have{a}{\neg S} \by{Premissa}{}
            \hypo{b}{\neg S \Rightarrow (P \land Q)} \by{Premissa}{}
            \have{c}{P \land Q} \ie{a,b}
            \have{d}{Q} \ae{c}
            \have{e}{Q \lor \neg R} \oi{d}
        \end{nd}
    $$
\end{exem}

%A regra de eliminação da disjunção é um pouco mais complicada, pois para ser realizada a mesma invoca duas sub-provas hipotéticas, formalmente tal regra é definida como se segue.

\begin{definition}[Regra de eliminação da disjunção ($\lor E$)]\label{def:EliminacaoDisjuncao}
    Sempre que existe uma palavra $\alpha \lor \beta$ na $i$-ésima linha da prova e for possível deduz $\gamma$ a partir de sub-provas iniciadas com $\alpha$ e $\beta$ como hipótese, então na linha $n$ tal que $i < n$ é possível deduzir a palavra $\gamma$. Na notação de diagramas tem-se que:
    $$
        \begin{nd}
            \have[i]{a}{\alpha \lor \beta}
            \have[\vdots]{skip1}{\vdots}
            \open
            \hypo[j]{b}{\alpha}
            \have[\vdots]{skip1}{\vdots}  
            \have[j + l_1]{c}{\gamma}
            \close
            \open
            \hypo[k]{d}{\beta}
            \have[\vdots]{skip1}{\vdots}  
            \have[k + l_2]{e}{\gamma}
            \close
            \have[\vdots]{skip1}{\vdots}  
            \have[n]{f}{\gamma} \oe{a, (b--c, d--e)}
        \end{nd}
    $$
\end{definition}

\begin{exem}\label{exe:EliminacaoDisjuncao}
     A prova de $\vdash ((P \lor Q) \land \neg P) \Rightarrow Q$ usa a regra de eliminação da disjunção.
     $$
        \begin{nd}
            \hypo{a}{(P \lor Q) \land \neg P} \by{Assuma}{}
            \have{b}{P \lor Q} \ae{a}
            \have{c}{\neg P} \ae{a}
            \open
            \hypo{d}{P} \by{Assuma}{}
            \open
            \hypo{e}{\neg Q} \by{Assuma}{}
            \have{f}{P} \by{REI}{d}
            \have{g}{\neg P} \by{REI}{c}
            \have{h}{\bot} \by{$\bot$ I}{f,g}
            \close
            \have{i}{\neg \neg Q} \ni{e--h}
            \have{j}{Q} \ne{i}
            \close
            \open
            \hypo{k}{Q} \by{Assuma}{}
            \have{l}{Q} \by{REI}{k}
            \close
            \have{m}{Q}  \oe{b, (d--j, k--l)}
            \close
            \have{n}{((P \lor Q) \land \neg P) \Rightarrow Q} \ii{a--m}
        \end{nd}
    $$
\end{exem}

Neste ponto do texto tem-se que foram apresentadas todas as regras básicas de inferência para o sistema de dedução natural da linguagem proposicional que foi apresentada na Definição \ref{def:LingProp}. Entretanto, pode-se pensar em extensões da linguagem proposicional baseado na ideia de abreviações (Definição \ref{def:SeSomenteSeAbreviatura}), assim é natural que os novos símbolos criados também possuam suas regras de introdução e eliminação, para o caso da bi-implicação que foi descrita neste texto (Definição \ref{def:SeSomenteSeAbreviatura}) são apresentadas as regras de introdução e eliminação como se seguem.

\begin{definition}[Regra de introdução da bi-implicação ($\Leftrightarrow I$)]\label{def:IntroducaoBicondicional}
   Se existe uma sub-prova iniciada na linha $i$ com $\alpha$ que conclui $\beta$ na linha $j$ e sub-prova iniciada na linha $k$ com $\beta$ que conclui $\alpha$ na linha $l$, então no escopo que contém as duas sub-provas  pode-se concluir $\alpha \Leftrightarrow \beta$ na linha $m$ com $i < j < k < l < m$. Em diagramas tem-se:
    
    $$
        \begin{nd}
            \open
            \hypo[i]{a}{\alpha}
            \have[\vdots]{skip1}{\vdots}
            \have[j]{b}{\beta}
            \close
            \open
            \hypo[k]{c}{\beta}
            \have[\vdots]{skip1}{\vdots}
            \have[l]{d}{\alpha}
            \close
            \have[m]{e}{\alpha \Leftrightarrow \beta}\by{$\Leftrightarrow I$}{a--b, c--d}
        \end{nd}
    $$
\end{definition}

\begin{definition}[regra de eliminação da bi-implicação ($\Leftrightarrow E$)]\label{def:EliminacaoBicondicional}
    Se em uma prova existe $\alpha \Leftrightarrow \beta$ em uma linha $i$, então é possível concluir $\alpha \Rightarrow \beta$ ou $\beta \Rightarrow \alpha$ em uma linha $j$ do diagrama com $i < j$. Na notação de diagramas de Fitch tem-se:
    
    \begin{minipage}{.40\textwidth} %
        $$
            \begin{nd}
                \have[\vdots]{skip1}{\vdots}  
                \have[i]{a}{\alpha \Leftrightarrow \beta}
                \have[\vdots]{skip1}{\vdots}  
                \have[j]{b}{\alpha \Rightarrow \beta}  \by{$\Leftrightarrow E$}{a} 
                \have[\vdots]{skip1}{\vdots} 
            \end{nd}
        $$
    \end{minipage} %
    ou
    \begin{minipage}{.40\textwidth} %
        $$
            \begin{nd}
                \have[\vdots]{skip1}{\vdots}  
                \have[i]{a}{\alpha \Leftrightarrow \beta}
                \have[\vdots]{skip1}{\vdots}  
                \have[j]{b}{\beta \Rightarrow \alpha}  \by{$\Leftrightarrow E$}{a} 
                \have[\vdots]{skip1}{\vdots}
            \end{nd}
        $$
    \end{minipage}
\end{definition}

\begin{rema}
    Uma vez que, não existe proibição de usar novamente uma palavra já obtida na prova (desde que seu escopo ainda esteja ativo), o leitor deve ficar atento ao fato de que a regra de eliminação da bi-implicação (Definição \ref{def:EliminacaoBicondicional}) não proibi que seja obtido as duas implicações, uma na linha $j$ e outra linha $j + k$ para algum $k \geq 1$. 
\end{rema}

\begin{exem}\label{exem:IntroducaoBiImplicacao}
     A prova de $ \vdash P \Leftrightarrow \neg \neg P$ é dado como se segue:
     $$
        \begin{nd}
            \open
            \hypo{a}{P} \by{Assuma}{}
            \open
            \hypo{b}{\neg P} \by{Assuma}{}
            \have{c}{P} \by{REI}{a}
            \have{d}{\neg P} \by{REI}{b}
            \have{e}{\bot} \by{$\bot$ I}{c,d}
            \close
            \have{f}{\neg \neg P} \ni{b--e}
            \close
            \open
            \hypo{g}{\neg \neg P} \by{Assuma}{}
            \have{h}{P} \ne{g}
            \close
            \have{i}{P \Leftrightarrow \neg \neg P}  \by{$\Leftrightarrow$I}{a--f, g--h} 
        \end{nd}
     $$
\end{exem}

\begin{exem}\label{exem:EliminacaoBiImplicacao}
     A prova de $\{P \Leftrightarrow Q, Q \land R \} \vdash P \land R$ é dado como se segue:
     $$
        \begin{nd}
            \have{a}{P \Leftrightarrow Q} \by{Premissa}{}
            \hypo{b}{Q \land R} \by{Premissa}{}
            \have{c}{Q} \ae{b}
            \have{d}{Q \Rightarrow P} \by{$\Leftrightarrow E$}{a} 
            \have{e}{P} \ie{c,d}
            \have{f}{R} \ae{b}
            \have{g}{P \land R} \ae{e,f}
        \end{nd}
     $$
\end{exem}

Para prosseguir serão apresentadas as propriedades do sistema dedutivo, os próximos resultados são meta-teoremas do sistema em si, isto é, são resultados  da natureza do sistema dedutivo em si e não palavras que podem ser deduzidas de um conjunto vazio de hipóteses.

\begin{theorem}[Teorema da dedução]\label{teo:TeoremaDeducaoSintatico}
    Seja $\Gamma \subseteq \mathcal{L}_{Prop}$ e $\alpha, \beta \in \mathcal{L}_{Prop}$. Se $\Gamma \cup \{\alpha\} \vdash \beta$, então $\Gamma \vdash \alpha \Rightarrow \beta$.
\end{theorem}

\begin{proof}
  Suponha que $\Gamma \cup \{\alpha\} \vdash \beta$, assim existe um diagrama da forma:
  $$
    \begin{nd}
        \have[1]{a}{\delta_1} \by{Premissa}{}
        \have[\vdots]{skip1}{\vdots} 
        \have[m]{b}{\delta_m} \by{Premissa}{}
        \hypo[m+1]{c}{\alpha} \by{Premissa}{}
        \have[\vdots]{skip1}{\vdots} 
        \have[n]{d}{\beta} 
    \end{nd}
  $$
  com $\delta_i \in \Gamma$ para todo $i \leq m$. Assim pode-se iniciar uma nova prova a partir das premissas $\delta_1, \cdots, \delta_m$ do conjunto $\Gamma$, em seguida é iniciada uma sub-prova hipotética tomando $\alpha$ como hipótese, ou seja, começa-se a desenvolver o seguinte diagrama:
  $$
    \begin{nd}
        \have[1]{a}{\delta_1} \by{Premissa}{}
        \have[\vdots]{skip1}{\vdots} 
        \hypo[m]{b}{\delta_m} \by{Premissa}{}
        \open
        \hypo[m+1]{c}{\alpha} \by{Assuma}{}
    \end{nd}
  $$
  Em seguida usando a regra REI é possível ``copiar'' todas as premissas para o escopo da sub-prova, atualizando o diagrama para a forma:
  $$
    \begin{nd}
        \have[1]{a}{\delta_1} \by{Premissa}{}
        \have[\vdots]{skip1}{\vdots} 
        \hypo[m]{b}{\delta_m} \by{Premissa}{}
        \open
        \hypo[m+1]{c}{\alpha} \by{Assuma}{}
        \have[m+1+1]{d}{\delta_1} \by{REI}{a}
        \have[\vdots]{skip1}{\vdots} 
        \have[2m+1]{e}{\delta_m} \by{REI}{b}
    \end{nd}
  $$
  Agora basta desenvolver a sub-prova utilizando exatamente a mesma sequência de regras utilizadas\footnote{Vale destacar que obviamente devem ser atualizadas as informações sobre as linhas de aplicação das regras, com respeito as linha do novo diagrama.} na prova original de $\Gamma \cup \{\alpha\} \vdash \beta$, e assim será possível deduzir a palavra $\beta$ na sub-prova após $n$ linhas depois da última premissa copiada com REI, ou seja, o diagrama fica com a seguinte forma:
  
  $$
    \begin{nd}
        \have[1]{a}{\delta_1} \by{Premissa}{}
        \have[\vdots]{skip1}{\vdots} 
        \hypo[m]{b}{\delta_m} \by{Premissa}{}
        \open
        \hypo[m+1]{c}{\alpha} \by{Assuma}{}
        \have[m+1+1]{d}{\delta_1} \by{REI}{a}
        \have[\vdots]{skip1}{\vdots} 
        \have[2m+1]{e}{\delta_m} \by{REI}{b}
        \have[\vdots]{skip1}{\vdots} 
        \have[2m+n+1]{f}{\beta}
    \end{nd}
  $$
  Portanto, utilizando a regra de introdução da implicação entre as linhas $m+1$ e $2m+n+1$, é obtida na linha $2m+n+2$ a palavra $\alpha \Rightarrow \beta$, ou seja, tem-se o seguinte diagrama:
  $$
    \begin{nd}
        \have[1]{a}{\delta_1} \by{Premissa}{}
        \have[\vdots]{skip1}{\vdots} 
        \hypo[m]{b}{\delta_m} \by{Premissa}{}
        \open
        \hypo[m+1]{c}{\alpha} \by{Assuma}{}
        \have[m+1+1]{d}{\delta_1} \by{REI}{a}
        \have[\vdots]{skip1}{\vdots} 
        \have[2m+1]{e}{\delta_m} \by{REI}{b}
        \have[\vdots]{skip1}{\vdots} 
        \have[2m+n+1]{f}{\beta}
        \close
        \have[2m+n+2]{g}{\alpha \Rightarrow \beta} \ii{c--f}
    \end{nd}
  $$
  O que mostra que, $\Gamma \vdash \alpha \Rightarrow \beta$, concluindo assim a prova.
\end{proof}

O resultado que se segue é uma consequência direta do Teorema da dedução.

\begin{corollary}\label{col:DeducaoSintatica}
    Seja $\Gamma \subseteq \mathcal{L}_{Prop}$ tal que $\Gamma = \{\alpha_1, \cdots, \alpha_n\}$ e $\beta \in \mathcal{L}_{Prop}$. Se $\Gamma \vdash \beta$, então $\vdash \alpha_1 \Rightarrow ( \cdots ( \alpha_n \Rightarrow \beta))$.
\end{corollary}

\begin{proof}
  A prova deste corolário consiste simplesmente de $n$ aplicações do Teorema da dedução (Teorema \ref{teo:TeoremaDeducaoSintatico}).
\end{proof}

\begin{theorem}\label{teo:TeoremaDeducaoSintatico2}
    Seja $\Gamma \subseteq \mathcal{L}_{Prop}$ e $\alpha, \beta \in \mathcal{L}_{Prop}$. Se $\Gamma \vdash \alpha \Rightarrow \beta$, então $\Gamma \cup \{\alpha\}  \vdash \beta$.
\end{theorem}

\begin{proof}
  A prova deste teorema apresenta uma simetria de raciocínio com o raciocínio apresentado pela prova do Teorema da dedução, assim sendo, a prova deste teorema irá ficar como exercício ao leitor.
\end{proof}

\begin{rema}
    Durante a prova do Teorema \ref{teo:TrasitividadeSintatica} a seguir, o símbolo $\Sigma$ irá denotar o somatório e não o alfabeto proposicional. 
\end{rema}

\begin{theorem}[Transitividade de $\vdash$]\label{teo:TrasitividadeSintatica}
    Seja $\Gamma \subseteq \mathcal{L}_{Prop}$ e $\alpha_1,\cdots,\alpha_n, \beta \in \mathcal{L}_{Prop}$. Se $\Gamma \vdash \alpha_1, \cdots, \Gamma \vdash \alpha_n$ e $\{\alpha_1, \cdots, \alpha_n\} \vdash \beta$, então $\Gamma \vdash \beta$
\end{theorem}

\begin{proof}
  Assuma que para todo $1 \leq i \leq n$ tem-se que $\Gamma \vdash \alpha_i$, ou seja, para cada $\alpha_i$ existe um diagrama da forma:
  $$
    \begin{nd}
        \have[1]{a}{\delta_1} \by{Premissa}{}
        \have[\vdots]{skip1}{\vdots} 
        \hypo[m]{b}{\delta_m} \by{Premissa}{}
        \have[\vdots]{skip1}{\vdots} 
        \have[k_i]{c}{\alpha_i}
    \end{nd}
  $$
  com $\delta_c \in \Gamma$ para todo $1 \leq c \leq m$ e $k_i \in \mathbb{N}$ tal que $k_i > m$. Além disso, assuma também que $\{\alpha_1, \cdots, \alpha_n\} \vdash \beta$, ou seja, assuma que existe um diagrama da forma:
  $$
    \begin{nd}
        \have[1]{a}{\alpha_1} \by{Premissa}{}
        \have[\vdots]{skip1}{\vdots} 
        \hypo[n]{b}{\alpha_n} \by{Premissa}{}
        \have[\vdots]{skip1}{\vdots} 
        \have[l]{c}{\beta}
    \end{nd}
  $$
  Assim é possível construir um novo diagrama em que as linhas de 1 até $m$ contém exatamente as premissas que formam o conjunto $\Gamma$. Além disso, as linhas de $m+1$ até $k_1$ serão idênticas as linhas do diagrama para a prova de $\Gamma \vdash \alpha_1$. Seguindo a construção deste novo diagrama para cada $2 \leq j \leq n$, as linha de $m+1$ até $k_j$ dos diagramas das prova de $\Gamma \vdash \alpha_j$ são dispostas sequencialmente no novo diagrama da mesma forma que estão em seus diagramas originais \footnote{Exceto pelo fato de que a referência das linhas de aplicação das regras de dedução natural precisam ser atualizadas para as linhas do novo diagrama}, assim o novo diagrama possui neste momento a seguinte forma:
  $$
    \begin{nd}
        \have[1]{a}{\delta_1} \by{Premissa}{}
        \have[\vdots]{skip1}{\vdots} 
        \hypo[m]{b}{\delta_m} \by{Premissa}{}
        \have[\vdots]{skip1}{\vdots} 
        \have[k_1]{c}{\alpha_1}
        \have[\vdots]{skip1}{\vdots} 
        \have[m + \sum_{i=1}^n (k_i - m)]{d}{\alpha_n}
    \end{nd}
  $$
  Aplicando então a regra REI $n$ vezes pode-se copiar as palavras $\alpha_1, \cdots, \alpha_n$ para as linhas de $m + \sum_{i=1}^n (k_i - m) + 1$ até $m + \sum_{i=1}^n (k_i - m) + n$, atualizando assim o novo diagrama para a forma:
  $$
    \begin{nd}
        \have[1]{a}{\delta_1} \by{Premissa}{}
        \have[\vdots]{skip1}{\vdots} 
        \hypo[m]{b}{\delta_m} \by{Premissa}{}
        \have[\vdots]{skip1}{\vdots} 
        \have[k_1]{c}{\alpha_1}
        \have[\vdots]{skip1}{\vdots} 
        \have[m + \sum_{i=1}^n (k_i - m)]{d}{\alpha_n}
        \have[\Big(m + \sum_{i=1}^n (k_i - m)\Big) + 1]{e}{\alpha_1} \by{REI}{c}
        \have[\vdots]{skip1}{\vdots} 
        \have[\Big(m + \sum_{i=1}^n (k_i - m)\Big) + n]{e}{\alpha_n} \by{REI}{d}
    \end{nd}
  $$
  Agora basta repetir no novo diagrama todas as linhas de $n$ até $l$ do diagrama da prova de $\{\alpha_1, \cdots, \alpha_n\} \vdash \beta$ exatamente como estão dispostas, atualizando apenas as referência das aplicações das regra de dedução natural, assim o diagrama assumi a forma:
  $$
    \begin{nd}
        \have[1]{a}{\delta_1} \by{Premissa}{}
        \have[\vdots]{skip1}{\vdots} 
        \hypo[m]{b}{\delta_m} \by{Premissa}{}
        \have[\vdots]{skip1}{\vdots} 
        \have[k_1]{c}{\alpha_1}
        \have[\vdots]{skip1}{\vdots} 
        \have[m + \sum_{i=1}^n (k_i - m)]{d}{\alpha_n}
        \have[\Big(m + \sum_{i=1}^n (k_i - m)\Big) + 1]{e}{\alpha_1} \by{REI}{c}
        \have[\vdots]{skip1}{\vdots} 
        \have[\Big(m + \sum_{i=1}^n (k_i - m)\Big) + n]{e}{\alpha_n} \by{REI}{d}
        \have[\vdots]{skip1}{\vdots} 
        \have[\Big(m + \sum_{i=1}^n (k_i - m)\Big) + n + l]{e}{\beta}
    \end{nd}
  $$
  E tal diagrama é exatamente a prova de $\Gamma \vdash \beta$, o que conclui a demonstração.
\end{proof}

Seguindo com a apresentação das propriedades do sistema dedutivo, a seguir serão exposto duas diferente formas de apresentar e demonstrar a noção de monotonicidade de $\vdash$.

\begin{theorem}[Teorema da monotonicidade (versão 1)]\label{teo:MonotonicidadeSintaticaV1}
    Sejam $\Gamma_1$ e $\Gamma_2$ dois subconjuntos de $\mathcal{L}_{Prop}$ e seja $\alpha \in \mathcal{L}_{Prop}$. Se $\Gamma_1 \vdash \alpha$, então $\Gamma_1 \cup \Gamma_2 \vdash \alpha$.
\end{theorem}

\begin{proof}
  Sem perda de generalidade assuma que $\Gamma_1 = \{\alpha_1, \cdots, \alpha_m\}$ e que $\Gamma_2 = \{\beta_1, \cdots, \beta_n\}$ com $\Gamma_1 \cap \Gamma_2 = \emptyset$. Agora suponha que $\Gamma_1 \vdash \alpha$, logo existe um diagrama na forma:
  $$
    \begin{nd}
        \have[1]{a}{\alpha_1} \by{Premissa}{}
        \have[\vdots]{skip1}{\vdots} 
        \hypo[m]{b}{\alpha_m} \by{Premissa}{}
        \have[\vdots]{skip1}{\vdots} 
        \have[k]{c}{\alpha}
    \end{nd}
  $$
  Agora é claro que é possível construir um novo diagrama em que além das premissas $\alpha_1, \cdots, \alpha_m$ são também usadas as premissas do conjunto $\Gamma_2$, de forma que este diagrama inicialmente seja da forma:
  $$
    \begin{nd}
        \have[1]{a}{\alpha_1} \by{Premissa}{}
        \have[\vdots]{skip1}{\vdots} 
        \have[m]{b}{\alpha_m} \by{Premissa}{}
        \have[m+1]{b}{\beta_1} \by{Premissa}{}
        \have[\vdots]{skip1}{\vdots}
        \hypo[m+n]{d}{\beta_n} \by{Premissa}{}
    \end{nd}
  $$
  Agora basta atualizar\footnote{A atualização deve indexar corretamente a aplicação das provas com respeito as linhas do novo diagrama.} este diagrama repetindo todas as linhas de $m+1$ até a linha $k$ do diagrama da prova de $\Gamma_1 \vdash \alpha$, e dessa forma o novo diagrama será atualizado para a forma:
  $$
    \begin{nd}
        \have[1]{a}{\alpha_1} \by{Premissa}{}
        \have[\vdots]{skip1}{\vdots} 
        \have[m]{b}{\alpha_m} \by{Premissa}{}
        \have[m+1]{b}{\beta_1} \by{Premissa}{}
        \have[\vdots]{skip1}{\vdots}
        \hypo[m+n]{d}{\beta_n} \by{Premissa}{}
        \have[\vdots]{skip1}{\vdots}
        \have[m+n+(k-m+1)]{e}{\alpha}
    \end{nd}
  $$
  Mas tal diagrama é exatamente uma prova de $\Gamma_1 \cup \Gamma_2 \vdash \alpha$.
\end{proof}

Para prosseguir é necessário antes considerar a definição de conjunto das palavras deduzíveis apresentada a seguir.

\begin{definition}
    Seja $\Gamma \subseteq \mathcal{L}_{Prop}$ o conjunto de todas as palavras deduzíveis de $\Gamma$, denotada por $Th(\Gamma)$, é o conjunto de todas as palavras que são consequência sintática de $\Gamma$, ou seja, $Th(\Gamma) = \{ \alpha \in \mathcal{L}_{Prop} \mid \Gamma \vdash \alpha\}$.
\end{definition}

\begin{theorem}
    Para qualquer que seja $\Gamma \subseteq \mathcal{L}_{Prop}$ tem-se que $Th(\Gamma)$ é infinito.
\end{theorem}

\begin{proof}
  Não é difícil verificar que $Th(\Gamma) \neq \emptyset$ para qualquer que seja $\Gamma \subseteq \mathcal{L}_{Prop}$\footnote{O leitor pode treinar seu raciocínio de construção de provas demonstrando essa afirmação.}, dessa forma existe pelo menos um $\alpha \in Th(\Gamma)$. Agora escolhendo qualquer $\beta \in \mathcal{L}_{Prop}$ considere agora uma palavra $\alpha \lor \beta$, e uma vez que, $\Gamma \vdash \alpha$, é claro pela regra de introdução da disjunção que $\Gamma \vdash \alpha \lor \beta$, e portanto, $\alpha \lor \beta \in Th(\Gamma)$. Uma vez que existem infinitos $\beta \in \mathcal{L}_{Prop}$ tem-se que $Th(\Gamma)$ é infinito.
\end{proof}

\begin{theorem}[Teorema da monotonicidade (versão 2)]\label{teo:MonotonicidadeSintaticaV2}
    Sejam $\Gamma_1$ e $\Gamma_2$ dois subconjuntos de $\mathcal{L}_{Prop}$. Se $\Gamma_1 \subset \Gamma_2$, então $Th(\Gamma_1) \subset Th(\Gamma_2)$.
\end{theorem}

\begin{proof}
  Suponha que $\Gamma_1 \subset \Gamma_2$, agora para qualquer $\alpha \in Th(\Gamma_1)$, uma vez que, $\Gamma_1 \vdash \alpha$ pelo Teorema \ref{teo:MonotonicidadeSintaticaV1} tem-se que $\Gamma_1 \cup \Gamma_2 \vdash \alpha$, entretanto, pelas propriedades da união de conjuntos, como $\Gamma_1 \subset \Gamma_2$ tem-se que $\Gamma_2 = \Gamma_1 \cup \Gamma_2$, e portanto, $\Gamma_2 \vdash \alpha$, consequentemente, $\alpha \in Th(\Gamma_2)$. E assim tem-se que $Th(\Gamma_1) \subset Th(\Gamma_2)$.
\end{proof}

\begin{theorem}[Teorema da compacidade]\label{teo:TeoremaDaCompacidade}
    $\Gamma \vdash \alpha$ se, e somente se, existe um conjunto finito $\Gamma_0 \subseteq \Gamma$ tal que $\Gamma_0 \vdash \alpha$.
\end{theorem}

\begin{proof}
  ($\Rightarrow$) Assuma que $\Gamma \vdash \alpha$ assim existe uma diagrama de Fitch que prova tal afirmação, mas uma vez que, um diagrama tem um número finito de linhas tem-se que existe $n$ premissas de $\Gamma$ usadas no diagrama, e portanto, existe um $\Gamma_0$ que é formado exatamente pelas premissas usadas, o que implica que $\Gamma_0$ é finito e $\Gamma_0 \vdash \alpha$, e além disso, por definição de subconjunto é claro que $\Gamma_0 \subseteq \Gamma$.
  
  ($\Leftarrow$) Suponha que $\Gamma_0 \subseteq \Gamma$ e que $\Gamma_0 \vdash \alpha$ pelo Teorema \ref{teo:MonotonicidadeSintaticaV2} tem-se que todo $\alpha \in Th(\Gamma_0)$ é tal que $\alpha \in Th(\Gamma)$, mas assim por definição $\Gamma \vdash \alpha$ o que completa a prova.
\end{proof}

\begin{theorem}[Teorema do ponto fixo]\label{teo:PontoFixo}
    $Th(\Gamma) = Th(Th(\Gamma))$ para qualquer $\Gamma$.
\end{theorem}

\begin{proof}
  Primeiro note que para qualquer $\alpha \in Th(\Gamma)$, uma vez que, usando a regra REI pode-se demonstrar que $\{\alpha\} \vdash \alpha$, dessa forma tem-se que $\alpha \in Th(Th(\Gamma))$, e portanto, $Th(\Gamma) \subset Th(Th(\Gamma))$. Por outro lado, suponha por absurdo que $Th(Th(\Gamma)) \not\subset Th(\Gamma)$, assim existe um $\alpha \in Th(Th(\Gamma))$ tal que $\alpha \not\in Th(\Gamma)$, mas por definição $\alpha$ necessariamente deve ser deduzidas de um conjunto de premissas $\{\beta_1, \cdots, \beta_n\} \subset Th(\Gamma)$, e nessas condições pelo Teorema \ref{teo:MonotonicidadeSintaticaV2} tem-se que $\alpha \in Th(\Gamma)$ o que contradiz a hipótese, e portanto, $Th(Th(\Gamma)) \subset Th(\Gamma)$. Agora uma vez que, $Th(\Gamma) \subset Th(Th(\Gamma))$ e $Th(Th(\Gamma)) \subset Th(\Gamma)$, tem-se que $Th(\Gamma) = Th(Th(\Gamma))$, completando assim a prova.
\end{proof}

\section{Sistema Axiomático}\label{sec:SistemaAxiomatico}

Uma abordagem alternativa para o sistema de dedução natural são os chamados sistemas axiomáticos\footnote{Em algumas referências como por exemplo \cite{BenjaV1, leonidas2002} é usado o termo teorias formais em vez de sistemas axiomáticos.}, esses sistemas introduzidos inicialmente pelo matemático alemão David Hilbert (1862-1943), consistem em adotar um conjunto finito de axiomas e um número reduzido de regras de inferência \cite{joaoPavao2014, sernadas2006}.

Antes de prosseguir para definir precisamente a noção de sistemas axiomáticos é necessário antes falar sobre provas para o sistemas axiomáticos.

\begin{definition}[Provas em sistemas axiomáticos]\label{def:ProvaAxiomatica}
    \cite{leonidas2002} Uma prova de uma palavra $\alpha$ em um sistema axiomático $T$ é uma tabela com $n$ linhas e $3$ colunas, em que, para cada linha da tabela: a primeira coluna informa o número da linha, a segunda coluna contém uma palavra da linguagem do sistema axiomático $T$ e a terceira coluna informa se a palavra contida na segunda coluna é um premissa, instância de axioma ou informa a regra de inferência e as linhas usadas para obter tal palavra. 
\end{definition}

Note que a Definição \ref{def:ProvaAxiomatica} menciona a noção de linguagem do sistema $T$, isso faz referência ao fato de que cada sistema axiomático considera apenas um subconjunto de conectivos lógicos apresentados na Definição \ref{def:AlfProp}, assim as palavras que ocorrem na linguagem do sistema axiomático é restrita as palavras com tais conectivos. Vale ressaltar que isso não diminui o poder de dedução dos sistemas axiomáticos, pois como explicado em \cite{BenjaV1}, existe uma relação entre os conectivos que torna sempre possível obter um usando os outros, isso é possível usando abreviações como descrito a seguir. 

\begin{definition}\label{def:Abreviacoes}
    Seja $\Sigma_s$ da mesma forma que apresentado na Definição \ref{def:AlfProp} para todo $\alpha, \beta \in \Sigma_s \cup \{\bot\}$ são estabelecidas as seguintes abreviações:
    \begin{itemize}
        \item $\alpha \land \beta \equiv_{abr} \neg (\neg \alpha \lor \neg \beta)$.
        \item $\alpha \lor \beta \equiv_{abr} \neg (\neg \alpha \land \neg \beta)$.
        \item $\alpha \Rightarrow \beta \equiv_{abr} \neg \alpha \lor \beta$.
        \item $\neg \alpha \equiv_{abr} \alpha \Rightarrow \bot$.
    \end{itemize}
\end{definition}

Assim as abreviações podem ser vistas como um homomorfismo\footnote{Aqui o homomorfismo preserva as relações de consequência, no sentido de que se em $\mathcal{L}_{Prop}$ existe uma prova de $\Gamma \vdash \alpha$ usando dedução natural, então existe uma prova em $\mathcal{L}_T$ de $Abr(\Gamma) \vdash_T Abr(\alpha)$ em um determinado sistema axiomático $T$ e sendo $Abr(\Gamma)$ e $Abr(\alpha)$ respectivamente a aplicação de abreviações sobre as palavras de $\Gamma$ e sobre a palavra $\alpha$. Deste fato, o usa de uma linguagem formal $\mathcal{L}_T$ e um sistema axiomático $T$ não prejudica a generalidade da lógica proposicional.} da linguagem $\mathcal{L}_{Prop}$ para a linguagem $\mathcal{L}_T$ de algum sistema axiomático $T$. A seguir são formalizados os conceitos de sistema axiomático e de consequência para o sistema axiomático.

\begin{definition}[Sistema axiomático]\label{def:SistemaAxiomatico}
    Um sistema axiomático $T$ é uma estrutura da forma $\langle \mathcal{L}_T, Axi, Reg \rangle$ onde:
    \begin{itemize}
        \item $\mathcal{L}_T$ é a linguagem do sistema axiomático.
        \item $Axi \subset \mathcal{L}_T$ é um conjunto finito de axiomas.
        \item $Reg$ é um conjunto de regras de inferência.
    \end{itemize}
\end{definition}

\begin{definition}[Consequência em sistemas axiomáticos]\label{def:ConsequenciaAxiomatica}
    Seja $T$ um sistema axiomático e seja $\Gamma \subseteq \mathcal{L}_T$ e $\alpha \in \mathcal{L}_T$, é dito que $\alpha$ é consequência sintática de $\Gamma$ no sistema $T$, denotado por $\Gamma \vdash_T \alpha$, se existir uma prova partindo das premissas em $\Gamma$ que deduza $\alpha$.
\end{definition}

A seguir é apresentado o sistema axiomático conhecido como sistema $L$ a mesma é definida sobre a linguagem $\mathcal{L}_\Rightarrow$, ou como também é conhecida linguagem implicativa.

\begin{definition}[Linguagem $\mathcal{L}_\Rightarrow$]\label{def:LinguagemL}
    Dado o conjunto $\Sigma_s \cup \{\bot\}$, a linguagem $\mathcal{L}_\Rightarrow$ é o menor conjunto indutivamente gerado pelas regras:
    \begin{itemize}
        \item[(a)] Se $\alpha \in \Sigma_s \cup \{\bot\}$, então $\alpha \in \mathcal{L}_\Rightarrow$.
        \item[(b)] Se $\alpha \in \mathcal{L}_\Rightarrow$, então $(\neg \alpha) \in \mathcal{L}_\Rightarrow$.
        \item[(c)] Se $\alpha, \beta \in \mathcal{L}_\Rightarrow$, então $(\alpha \Rightarrow \beta) \in \mathcal{L}_\Rightarrow$.
    \end{itemize}
\end{definition}

\begin{definition}[Sistema $L$]\label{def:SistemaL}
   \cite{mendelson2009livro} O sistema $L$ é a  estrutura $\langle \mathcal{L}_\Rightarrow, Axi, \{ E\Rightarrow\}\rangle$ onde $Axi$ é o conjunto formado para todo $\alpha, \beta, \gamma \in \mathcal{L}_\Rightarrow$ pelos seguintes axiomas:
\end{definition}

\begin{itemize}
    \item[($A_1$)] $\alpha \Rightarrow (\beta \Rightarrow \alpha)$.
    \item[($A_2$)] $(\alpha \Rightarrow (\beta \Rightarrow \gamma)) \Rightarrow ((\alpha \Rightarrow \beta) \Rightarrow (\alpha \Rightarrow \gamma))$.
    \item[($A_3$)] $(\neg \beta \Rightarrow \neg \alpha) \Rightarrow ((\neg \beta \Rightarrow \alpha) \Rightarrow \beta)$.
\end{itemize}

\begin{exem}
    Uma prova de $ \vdash_L P \Rightarrow P$ é dada por:
    \begin{table*}[ht]
        \centering
        \begin{tabular}{|c|c|c|}
            \hline
            1 & $P \Rightarrow ((P \Rightarrow P) \Rightarrow P)$ & $A_1$\\
            \hline
            2 & $(P \Rightarrow ((P \Rightarrow P) \Rightarrow P)) \Rightarrow ((P \Rightarrow (P \Rightarrow P)) \Rightarrow (P \Rightarrow P))$ & $A_2$\\
            \hline
            3 & $P \Rightarrow (P \Rightarrow P)$ & $A_1$\\
            \hline
            4 & $(P \Rightarrow (P \Rightarrow P)) \Rightarrow (P \Rightarrow P)$ & $ E\Rightarrow  \ 1, 2$\\
            \hline
            5 & $P \Rightarrow P$ & $E\Rightarrow \ 3, 4$\\
            \hline
        \end{tabular}
    \end{table*}
\end{exem}

\begin{exem}
    Uma prova de $\{\neg P \Rightarrow \neg P, \neg P \Rightarrow P\} \vdash_L P$ é dada por:
    \begin{table*}[ht]
        \centering
        \begin{tabular}{|c|c|c|}
            \hline
            1 & $\neg P \Rightarrow \neg P$ & Premissa \\
            \hline
            2 & $\neg P \Rightarrow P$ & Premissa\\
            \hline
            3 & $(\neg P \Rightarrow \neg P) \Rightarrow ((\neg P \Rightarrow P) \Rightarrow P)$ & $A_3$\\
            \hline
            4 & $(\neg P \Rightarrow P) \Rightarrow P$ & $E\Rightarrow$ 1,3\\
            \hline
            5 & $P$ & $E\Rightarrow$ 2,4\\
            \hline
        \end{tabular}
    \end{table*}
\end{exem}

\begin{rema}\label{rema:EquivalenciaDeducaoNaturalAxiomatico}
    O Teorema da dedução (Teorema \ref{teo:TeoremaDeducaoSintatico}) e os demais meta-teoremas provados na seção anterior para linguagem $\mathcal{L}_{Prop}$ usando a dedução natural são válidos para qualquer sistema axiomático $T$ e sua linguagem \cite{BenjaV1}.
\end{rema}

\begin{exem}
    Uma prova de $\{P \Rightarrow Q, Q \Rightarrow R, P\} \vdash_L R$ é detalhada a seguir:
    \begin{table*}[ht]
        \centering
        \begin{tabular}{|c|c|c|}
            \hline
            1 & $P \Rightarrow Q$ & Premissa \\
            \hline
            2 & $Q \Rightarrow R$ & Premissa \\
            \hline
            3 & $P$ & Premissa \\
            \hline
            4 & $Q$ & $E\Rightarrow$ 1,3 \\
            \hline
            5 & $R$ & $E\Rightarrow$ 2,4 \\
            \hline
        \end{tabular}
    \end{table*}
    
    Agora pela prova acima e pelo teorema da dedução (no sistema $L$) tem-se que $\{P \Rightarrow Q, Q \Rightarrow R\} \vdash_L P \Rightarrow R$.
\end{exem}

\begin{rema}
    Todo sistema axiomático permite o uso de hipótese adicionais as premissas durante uma demonstração, ver Exemplo 4.3.2 de \cite{BenjaV1}.
\end{rema}

Outro importante sistema axiomático sobre a linguagem implicativa é o sistema $M$ de Meridith \cite{meredith1954}. Já o matemático David Hilbert e seu aluno Wilhelm Ackermann (1896-1962) no livro \cite{hilbert1999}, apresentam um sistema axiomático sobre a linguagem $\mathcal{L}_\lor$ descrita em \cite{BenjaV1}. Além desses diversos outros sistemas axiomáticos podem ser consultados em \cite{BenjaV1}.

\section{Sistema Semântico}\label{sec:SistemaSemantico}

A semântica da lógica proposicional foi descrita inicialmente pelo matemático inglês George Boole (1815-1864) em seu trabalho \cite{boole1854, boole1957}, entretanto, uma formulação mais rigorosa para computar os valores lógicos das palavras da linguagem proposicional foi apresenta somente em 1936 pelo lógico, matemático e filósofo polonês Alfred Tarski\footnote{O artigo de Tarski pode ser consultado na versão re-editada em \cite{tarski1983}.} (1901-1983). 

A semântica é responsável por introduzir significado para as palavras de uma linguagem formal, no caso da linguagem proposicional clássica\footnote{Clássica aqui diz respeito a lógica como apresentada pelo  matemático, lógico e filósofo alemão Gottlob Frege (1848-1925) no final do século 19.}, as palavras podem ter seu significado como verdadeiro ou falso. Antes de apresentar formalmente o conceito de semântica da linguagem proposicional é necessário definir a ideia de função de  valoração.

\begin{definition}[Valoração]
   Uma valoração dos símbolos proposicionais é uma função $\rho : \Sigma_s \rightarrow \{0,1\}$.
\end{definition}

\begin{rema}
    Uma vez que este texto é destinado prioritariamente a alunos de ciência exatas e tecnológicas aqui a imagem de qualquer valoração será o conjunto $\{0,1\}$ onde $0$ irá reapresentar o significado de \textbf{falso} e 1 irá representar o significado de \textbf{verdadeiro}.
\end{rema}

A semântica da linguagem proposicional como destacado em \cite{joaoPavao2014} se baseia na noção de interpretação\footnote{A definição de semântica apresentada neste texto é uma definição equacional, no sentido de que, sempre existe uma equação que determina o valor semântico para qualquer que seja a palavra de $\mathcal{L}_{Prop}$.}, sendo que, uma interpretação nada mais é do que a extensão de uma dada valoração $\rho$ para a linguagem proposicional.

\begin{definition}[Interpretação]\label{def:interpretacat}
   Dada uma valoração $\rho$, uma interpretação é uma função $I_\rho : \mathcal{L}_{Prop} \rightarrow \{0,1\}$ definida\footnote{O símbolo $\oplus$ denota a soma limitada, isto é, $\oplus$ se comporta com a soma convencional para todos os valores $x, y \in \{0,1\}$ tal que $x + y \leq 1$ e para o caso de $x + y > 1$ tem-se que $x \oplus y = 1$. As demais operações são as operações padrão e ordem usual do conjunto $\mathbb{Z}_+$.} para todo $\alpha, \beta \in \mathcal{L}_{prop}$ recursivamente como:
   \begin{itemize}
        \item Se $\alpha = \bot$, então $I_\rho(\alpha) = 0$.
        \item Se $\alpha \in \Sigma_s$, então $I_\rho(\alpha) = \rho(\alpha)$.
        \item $I_\rho(\neg \alpha) = 1 - I_\rho(\alpha)$.
        \item $I_\rho(\alpha \land \beta) = I_\rho(\alpha) \cdot I_\rho(\beta)$.
        \item $I_\rho(\alpha \lor \beta) = I_\rho(\alpha) \oplus I_\rho(\beta)$.
        \item $I_\rho(\alpha \Rightarrow \beta) = \left\{\begin{array}{rl}	1, & \hbox{se } I_\rho(\alpha) \leq I_\rho(\beta)\\	0,  & \hbox{senão}\end{array}\right.$.
   \end{itemize}
\end{definition}

\begin{rema}
    Para o caso do leitor estar considerando uma linguagem enriquecida pela símbolo de bi-implicação, tem-se que a semântica de tal operador é definida como:
    \begin{equation*}
        I_\rho(\alpha \Leftrightarrow \beta) = \left\{\begin{array}{rl}	1, & \hbox{se } I_\rho(\alpha) = I_\rho(\beta)\\	0,  & \hbox{senão}\end{array}\right.
    \end{equation*}
    para todo $\alpha, \beta \in \mathcal{L}_{Prop}$.
\end{rema}

\begin{exem}
    Considere uma valoração $\rho$ tal que $\rho(P) = 0, \rho(Q) = 1$ e $\rho(R) = 1$ o valor de significado da palavra $\neg(P \Rightarrow (R \land Q))$ é calculado por:
    \begin{equation}\label{eq:ExemploValoracaoA}
        I_\rho(\neg(P \Rightarrow (R \land Q))) = 1 - I_\rho(P \Rightarrow (R \land Q))
    \end{equation}
    mas pela Definição \ref{def:interpretacat} para calcular $I_\rho(P \Rightarrow (R \land Q))$ deve-se antes calcular $I_\rho(P)$ e $I_\rho(R \land Q)$, mas por definição tem-se que:
    \begin{eqnarray}\label{eq:ExemploValoracaoB}
        I_\rho(P) & = & \rho(P) = 1
    \end{eqnarray}
    e 
    \begin{eqnarray}\label{eq:ExemploValoracaoC}
        I_\rho(R \land Q) & = & I_\rho(R) \cdot I_\rho(Q) \nonumber\\
        & = & \rho(R) \cdot \rho(Q) \\
        & = & 1 \cdot 1 \nonumber \\
        & = & 1\nonumber
    \end{eqnarray}
    usando então a Definição \ref{def:interpretacat} e o valores das Equações (\ref{eq:ExemploValoracaoB}) e (\ref{eq:ExemploValoracaoC}) tem-se que $I_\rho(P \Rightarrow (R \land Q)) = 1$, substituindo esse valor na Equação (\ref{eq:ExemploValoracaoA}) é obtido que:
    \begin{eqnarray*}
        I_\rho(\neg(P \Rightarrow (R \land Q))) = 1 - 1 = 0
    \end{eqnarray*}
    isto é, a valoração da palavra $\neg(P \Rightarrow (R \land Q))$ é igual a 0, ou seja, a mesma é falsa.
\end{exem}

Pode-se também calcular a interpretação das palavras $\mathcal{L}_{Prop}$ usando a ideia de tabelas verdades, cada conectivo possui sua própria lei de formação de tabelas verdade \cite{BenjaV1, joaoPavao2014}, sendo que essas leis são derivadas das equações da Definição \ref{def:interpretacat}. Antes de prosseguir para explicar a construção das tabelas verdade é conveniente introduzir alguns conceitos chaves.

\begin{definition}[Conjunto de sub-palavras]
   Seja $\alpha \in \mathcal{L}_{Prop}$ o conjunto das sub-palavras de $\alpha$, denotado por $Sub_{\alpha}$, é o menor conjunto recursivamente gerado pelas seguintes regras:
   \begin{itemize}
        \item[R1.] Se $\alpha \in \Sigma_s \cup \{\bot\}$, então $Sub_\alpha = \{\alpha\}$.
        \item[R2.] Se $\alpha = \neg \beta$ com $\beta \in \mathcal{L}_{Prop}$, então $\neg \beta, \beta \in Sub_\alpha$ e todo $\alpha_i \in Sub_\beta$ é tal que $\alpha_i \in Sub_\alpha$.
        \item[R3.] Se $\alpha = \beta \bullet \gamma$  com $\bullet \in \{\land, \lor, \Rightarrow\}$ e $\beta, \gamma \in \mathcal{L}_{Prop}$, então $\beta \bullet \gamma, \beta, \gamma \in Sub_\alpha$ e todo $\alpha_i \in Sub_\beta \cup Sub_\gamma$ é tal que $\alpha_i \in Sub_\alpha$.
   \end{itemize}
\end{definition}

\begin{definition}[Átomos]
   Todo $\alpha \in \mathcal{L}_{Prop}$ tal que $\alpha \in \Sigma_s \cup \{\bot\}$ é chamado de átomo.
\end{definition}

\begin{definition}[Conjunto dos átomos]
   Seja $\alpha \in \mathcal{L}_{Prop}$ o conjunto dos átomos de $\alpha$  corresponde ao conjunto $Ato_\alpha = Sub_\alpha \cap (\Sigma_s \cup \{\bot\})$. 
\end{definition}

\begin{exem}
    Dado a palavra $P \Rightarrow (\neg Q \Rightarrow (P \lor T))$ tem-se:
    \begin{itemize}
        \item[(a)] As sub-palavras de $P \Rightarrow (\neg Q \Rightarrow (P \lor T))$ formam o conjunto:
        $$\{P, Q, T, \neg Q, P \lor T, \neg Q \Rightarrow (P \lor T), P \Rightarrow (\neg Q \Rightarrow (P \lor T))\}$$
        \item[(b)] Os átomos de $P \Rightarrow (\neg Q \Rightarrow (P \lor T))$ formam o conjunto:
        $$\{P, Q, T\}$$
    \end{itemize}
\end{exem}

Para realizar a construção da tabela verdade de uma fórmula $\alpha$, deve-se indexar as colunas da tabela com os elementos de $Sub_\alpha$, assim uma tabela verdade para uma fórmula $\alpha$ terá exatamente $\# Sub_\alpha$ colunas. Como os elementos de $Sub_\alpha$ não possuem uma ordenação, em geral é usado a regra de distribuir as palavras da esquerda para à direita de forma crescente de acordo com o tamanho das mesmas, assim as palavras mais à esquerda sempre terão tamanho 1, enquanto que a palavra mais à direita será sempre a maior palavra, isto é, a própria palavra $\alpha$.  A quantidade de linhas da tabela\footnote{Excluído a linha com os rótulos das colunas da tabela.}, por sua vez, é igual a $2^{\# Ato_\alpha}$, em que cada linha representa a aplicação de uma interpretação $I_{\rho_i}$ sobre a palavra $\alpha$.

Uma vez que os $\# Ato_\alpha$ átomos da palavra $\alpha$ sempre são as $\# Ato_\alpha$ primeira colunas da tabela verdade de $\alpha$, neste texto será considerado que as interpretações $I_{\rho_i}$, ou seja, as linhas da tabela, serão organizadas de forma a seguir a ordem lexicográfica das palavras do código binário com palavras de tamanho tamanho $\# Ato_\alpha$ com respeito as $\# Ato_\alpha$ primeiras colunas. 

\begin{exem}
    Para uma palavra cujo conjunto de átomos corresponde a $\{P, Q, S\}$ tem-se as seguintes linhas para as três primeiras colunas:
    \begin{table*}[ht]
        \centering
        \scriptsize
        \begin{tabular}{|c|c|c|c}
             \hline
             $P$ & $Q$ & $S$ & $\cdots$ \\
             \hline
             0 & 0 & 0 & $\cdots$ \\
             \hline
             0 & 0 & 1 & $\cdots$ \\
             \hline
             0 & 1 & 0 & $\cdots$ \\
             \hline
             0 & 1 & 1 & $\cdots$ \\
             \hline
             1 & 0 & 0 & $\cdots$ \\
             \hline
             1 & 0 & 1 & $\cdots$ \\
             \hline
             1 & 1 & 0 & $\cdots$ \\
             \hline
             1 & 1 & 1 & $\cdots$ \\
             \hline
        \end{tabular}
    \end{table*}
    Note que como explicado anteriormente as valorações (as linhas) estão ordenada seguinte a ordem lexicográfica do código binário, neste exemplo, é considerado o código binário com palavras de tamanho $3$.
\end{exem}
    
Para cada coluna que não seja rotulada por um átomo, o valor das linhas dessa coluna devem ser calculados usando as equações na Definição \ref{def:interpretacat}.

\begin{exem}
    Considerando a palavra $P \Rightarrow (\neg Q \Rightarrow (P \lor T))$ do exemplo anterior, tem-se a seguinte tabela verdade:
    \begin{table*}[ht]
        \centering
        \scriptsize
        \begin{tabular}{|c|c|c|c|c|c|c|}
             \hline
             $P$ & $Q$ & $T$ & $\neg Q$ & $P \lor T$ & $\neg Q \Rightarrow (P \lor T)$ & $P \Rightarrow (\neg Q \Rightarrow (P \lor T))$  \\
             \hline
             0 & 0 & 0 & 1 & 0 & 0 & 1\\
             \hline
             0 & 0 & 1 & 1 & 1 & 1 & 1\\
             \hline
             0 & 1 & 0 & 0 & 0 & 1 & 1\\
             \hline
             0 & 1 & 1 & 0 & 1 & 1 & 1\\
             \hline
             1 & 0 & 0 & 1 & 1 & 1 & 1\\
             \hline
             1 & 0 & 1 & 1 & 1 & 1 & 1\\
             \hline
             1 & 1 & 0 & 0 & 1 & 1 & 1\\
             \hline
             1 & 1 & 1 & 0 & 1 & 1 & 1\\
             \hline
        \end{tabular}
    \end{table*}
\end{exem}

\begin{exem}
    Para a palavra $P \Rightarrow \neg P \land Q$ tem-se a seguinte tabela verdade.
    \begin{table*}[ht]
        \centering
        \scriptsize
        \begin{tabular}{|c|c|c|c|c|}
             \hline
             $P$ & $Q$ & $\neg P$ & $\neg P \land Q$ & $P \Rightarrow \neg P \land Q$ \\ \hline
             0 & 0 & 1 & 0 & 1 \\ \hline
             0 & 1 & 1 & 1 & 1 \\ \hline
             1 & 0 & 0 & 0 & 0 \\ \hline
             1 & 1 & 0 & 0 & 0 \\ \hline
        \end{tabular}
    \end{table*}
\end{exem}

Obviamente o algoritmo responsável que usamos acima para a construção das tabelas verdades não é muito eficiente, de fato o mesmo tem uma complexidade exponencial na ordem de $2^{\#Ato_\alpha} \cdot \# Sub_\alpha$.

\begin{rema}
    Em \cite{leonidas2002}, é apresentado um método mais eficiente para a construção de tabelas verdade com tamanho reduzido, as chamadas tabelas abreviadas.     
\end{rema}

Como muito bem exposto em \cite{edgar2002, nunes2008}, as propriedades semânticas das palavras de $\mathcal{L}_{Prop}$, permitem construir três categorias ou classes, apresentadas a seguir

\begin{definition}[Tautologia]
    Uma palavra $\alpha \in \mathcal{L}_{Prop}$ é uma tautologia quando para toda interpretação $I_\rho$ tem-se que $I_\rho(\alpha) = 1$.
\end{definition}

\begin{definition}[Contradição]
    Uma palavra $\alpha \in \mathcal{L}_{Prop}$ é uma contradição quando para toda interpretação $I_\rho$ tem-se que $I_\rho(\alpha) = 0$.
\end{definition}

\begin{exem}\label{exe:TautologiaContradicao}
    As palavras $P \lor \neg P$  e $P \land \neg P$ são respectivamente uma tautologia e uma contradição, como pode ser visto nas tabelas abaixo.
    \begin{table*}[ht]
        \centering
        \scriptsize
        \begin{tabular}{cccc}
            \begin{tabular}{|c|c|c|}
                 \hline
                 $P$ & $\neg P$ & $P \lor \neg P$ \\ \hline
                 0 & 1 & 1\\ \hline
                 1 & 0 & 1\\ \hline
            \end{tabular}
            & &
            \begin{tabular}{|c|c|c|}
                 \hline
                 $P$ & $\neg P$ & $P \land \neg P$ \\ \hline
                 0 & 1 & 0\\ \hline
                 1 & 0 & 0\\ \hline
            \end{tabular}
        \end{tabular}
    \end{table*}
\end{exem}

\begin{prop}
    Para todo $\alpha \in \mathcal{L}_{prop}$ tem-se que $\alpha$ é uma tautologia se e somente se $\neg \alpha$ é uma contradição.
\end{prop}

\begin{proof}
  Trivial.
\end{proof}

\begin{definition}[Contingência]
    Uma palavra $\alpha \in \mathcal{L}_{Prop}$ é uma contingência quando não é uma tautologia e nem uma contradição.
\end{definition}

Usando a noção de interpretação é possível definir formalmente a noção de palavra satisfatível.

\begin{definition}[Palavra satisfatível]
    Uma palavra $\alpha \in \mathcal{L}_{Prop}$ é dita satisfatível quando existe uma interpretação $I_\rho$ tal que $I_\rho(\alpha) = 1$.
\end{definition}

\begin{exem}
    A palavra $P \Rightarrow Q$ é satisfatível pois existe uma interpretação $I_\rho$ com a valoração $\rho$ definido $\rho(Q) = 1$, e é claro pela Definição \ref{def:interpretacat} que tal interpretação torna o significado da palavra $P \Rightarrow Q$ como sendo verdadeiro, isto é, $I_\rho(P \Rightarrow Q) = 1$.
\end{exem}

\begin{definition}[Conjunto satisfatível]
    Um conjunto $\Gamma \subseteq \mathcal{L}_{Prop}$ é dito satisfatível se existe uma interpretação $I_\rho$ tal que para todo $\alpha \in \Gamma$ tem-se que $I_\rho(\alpha) = 1$.
\end{definition}

\begin{exem}
    O conjunto $\Gamma = \{P, Q, \neg P \lor Q\}$ é satisfatível pois a interpretação $I_\rho$ tal que $I_\rho(P) = 1$ e $I_\rho(Q) = 1$ é capaz de satisfazer todas as palavras de $\Gamma$.
\end{exem}

\begin{definition}[Conjunto Contraditório]
    Um conjunto $\Gamma \subseteq \mathcal{L}_{Prop}$ é dito contraditório se não existe nenhuma interpretação $I_\rho$ que satisfaça $\Gamma$.
\end{definition}

\begin{exem}
    O Conjunto $\Gamma = \{\neg \neg P, \neg P\}$ é contraditório, pois não existe nenhuma interpretação\footnote{Provar essa afirmação é um ótimo exercício para o leitor.} capaz de satisfazer todas as palavras em $\Gamma$.
\end{exem}

\begin{definition}[Modelo]
    Seja $\Gamma \subseteq \mathcal{L}_{Prop}$, se $\Gamma$ é satisfatível para uma certa interpretação $I_\rho$. Então é dito que $I_\rho$ é um modelo para $\Gamma$.
\end{definition}

Pode-se então agora formalizar a ideia de consequência do ponto de vista semântico, a consequência semântica é o mecanismo que determina se uma conclusão (ou tese) segue (ou é consequência) de um conjunto de premissas, formalmente isto é definido como se segue.

\begin{definition}[Consequência Semântica]\label{def:ConsequenciaSemantica}
    Seja $\Gamma \subseteq \mathcal{L}_{Prop}$ e $\alpha \in \mathcal{L}_{Prop}$ é dito que $\alpha$ é consequência semântica de $\Gamma$, denotado por $\Gamma \vDash \alpha$, se todo modelo $I_\rho$ de $\Gamma$ é também um modelo para o conjunto unitário $\{\alpha\}$.
\end{definition}

\begin{rema}\label{rema:ConsequenciaSemantica}
    Quando $\alpha$ é uma tautologia verifica-se que $\emptyset \vDash \alpha$, isto é, $\alpha$ é consequência semântica do conjunto vazio. Este fato também pode ser denotado por $\vDash \alpha$.
\end{rema}

\begin{exem}
    Seja $\Gamma = \{P, \neg P \lor Q \}$ tem-se que $\Gamma \vDash Q \lor \neg Q$, para verificar isso primeiro considere a seguinte tabela verdade\footnote{Aqui as tabelas verdades das palavras $\neg P$ e $\neg P \lor Q$ foram claramente concatenadas e reorganizadas.}:
    \begin{table*}[ht]
        \centering
        \scriptsize
        \begin{tabular}{|c|c|c|c|c|c|}
             \hline
             $P$ & $Q$ & $\neg P$ & $\neg P \lor Q $ \\ \hline
             0 & 0 & 1 & 1\\ \hline
             0 & 1 & 1 & 1\\  \hline
             1 & 0 & 0 & 0\\  \hline
             \rowcolor{cinzaClaro}
             1 & 1 & 0 & 1\\  \hline
        \end{tabular}
    \end{table*}
    
    Note que, a linha em destaque é o \textbf{único} modelo de $\Gamma$, isto é, tem-se que o modelo de $\Gamma$ é a interpretação $I_\rho$ tal que  $I_\rho(P) = 1$ e $I_\rho(Q) = 1$, pois com essa interpretação todas as palavras do conjunto $\Gamma$ são satisfeitas. Agora assumindo esse mesmo modelo tem-se que: 
    \begin{table*}[ht]
        \centering
        \scriptsize
        \begin{tabular}{|c|c|c|}
             \hline
             $Q$ & $\neg Q$ & $Q \lor \neg Q$ \\ \hline
             1 & 0 & 1 \\ \hline
        \end{tabular}
    \end{table*}
    
    E assim $\{Q \lor \neg Q\}$ também é satisfeito pelo mesmo modelo, como pode ser observado pela tabela verdade acima. Assim de fato tem-se que $\Gamma \vDash Q \lor \neg Q$.
\end{exem}

\begin{exem}
    Dado o conjunto $\Gamma = \{P, Q \lor R\}$ tem-se que $\neg P \land R$ não é consequência semântica de $\Gamma$, denotado por $\Gamma \not\vDash \neg P \land R$, pois se $\Gamma$ é satisfatível para $I_\rho$ tem-se que $I_\rho(P) = 1$, mas isso implicada que $I_\rho(\neg P) = 0$, e portanto, $I_\rho(\neg P \land R) = 0$, consequentemente, $\Gamma \not\vDash \neg P \land R$.
\end{exem}

\begin{rema}
    Como para a consequência sintática a consequência semântica pode ser vista como uma relação no sentido usual da teoria dos conjunto, ou seja, tem-se que $\vDash \subseteq \wp(\mathcal{L}_{Prop}) \times \mathcal{L}_{Prop}$.
\end{rema}

\begin{theorem}[Teorema da refutação]\label{teo:TeoremaRefutacao}
    Seja $\Gamma \subseteq \mathcal{L}_{Prop}$ e $\alpha \in \mathcal{L}_{Prop}$. $\Gamma \vDash \alpha$ se, e somente se, $\Gamma \cup \{\neg \alpha\}$ não é satisfatível. 
\end{theorem}

\begin{proof}
    ($\Rightarrow$) Suponha que  $\Gamma \vDash \alpha$ acontece, assim para todo $I_\rho$ que satisfaz $\Gamma$ tem-se que $I_\rho(\alpha) = 1$, mas pelo Definição \ref{def:interpretacat} tem-se que $I_\rho(\neg \alpha) = 0$, e portanto, nenhum $I_\rho$ será capaz de satisfazer $\Gamma \cup \{\neg \alpha\}$. 
    
    ($\Leftarrow$) Suponha que $\Gamma \cup \{\neg \alpha\}$ não é satisfatível, assim para qualquer interpretação $I_\rho$ que satisfaça $\Gamma$ obrigatoriamente não pode satisfazer $\{\neg \alpha\}$, ou seja, $I_\rho(\neg \alpha) = 0$, mas pela Definição \ref{def:interpretacat} tem-se que $I_\rho( \alpha) = 1$, e portanto, $\Gamma \vDash \alpha$.
\end{proof}

\begin{theorem}[Teorema da dedução (semântico)]\label{teo:TeoremaDeducaoSemantico}
    Para todo $\alpha, \beta \in \mathcal{L}_{Prop}$ tem-se que $\{\alpha\} \vDash \beta$ se, e somente se,  $\vDash \alpha \Rightarrow \beta$.
\end{theorem}

\begin{proof}
    ($\Rightarrow$) Suponha que $\{\alpha\} \vDash \beta$, assim para toda interpretação $I_\rho$ tal que $I_\rho(\alpha) = 1$ tem-se que $I_\rho(\beta) = 1$, logo $I_\rho(\alpha) \leq I_\rho(\beta)$, consequentemente, por definição tem-se que $I_\rho(\alpha \Rightarrow \beta) = 1$, e portanto, é modelo de $\{\alpha \Rightarrow \beta\}$, assim por vacuidade é claro que $\vDash \alpha \Rightarrow \beta$.
    
    ($\Leftarrow$) Suponha que $\vDash \alpha \Rightarrow \beta$, assim existe um modelo $I_\rho$ tal que $I_\rho(\alpha \Rightarrow \beta) = 1$, de fato existe $I_\rho(\alpha) = I_\rho(\beta) = 1$, mas por esta interpretação é também modelo para $\{\alpha\}$ e $\{\beta\}$, portanto, conclui-se que  $\{\alpha\} \vDash \beta$.
\end{proof}

\begin{corollary}\label{col:TeoremaDeducaoSemantico}
    Para todo $\alpha_1, \cdots, \alpha_{n-1}, \alpha_n, \beta \in \mathcal{L}_{Prop}$ tem-se que $\{\alpha_1, \cdots, \alpha_{n-1}, \alpha_n\} \vDash \beta$ se, e somente se, $\{\alpha_1, \cdots, \alpha_{n-1}\} \vDash \alpha_n \Rightarrow \beta$.
\end{corollary}

\begin{proof}
     ($\Rightarrow$) Suponha que $\{\alpha_1, \cdots, \alpha_{n-1}, \alpha_n\} \vDash \beta$, assim existe um modelo $I_\rho$ para $\{\alpha_1, \cdots, \alpha_{n-1}, \alpha_n\}$ e $\{\beta\}$, logo por definição tem-se para todo $i \leq n$ que $I_\rho(\alpha_i) = I_\rho(\beta) = 1$, consequentemente $I_\rho(\alpha_n) \leq I_\rho(\beta)$, dessa forma tem-se que $I_\rho(\alpha_n \Rightarrow \beta) = 1$, assim $I_\rho$ é um modelo para $\{\alpha_n \Rightarrow \beta\}$, e portanto, $\{\alpha_1, \cdots, \alpha_{n-1}\} \vDash \alpha_n \Rightarrow \beta$.
     
     ($\Leftarrow$) Suponha que  $\{\alpha_1, \cdots, \alpha_{n-1}\} \vDash \alpha_n \Rightarrow \beta$ logo existe um modelo $I_\rho$ que satisfaz $\{\alpha_1, \cdots, \alpha_{n-1}\} $ e $\{\alpha_n \Rightarrow \beta\}$, assuma que este modelo é tal que $I_\rho(\alpha_n) = I_\rho(\beta) = 1$, assim claramente tem-se que este modelo satisfaz $\{\alpha_1, \cdots, \alpha_{n-1}, \alpha_n\}$ e também satisfaz $\{\beta\}$, portanto, $\{\alpha_1, \cdots, \alpha_{n-1}, \alpha_n\} \vDash \beta$.
\end{proof}

\begin{prop}\label{prop:DeducaoSemantica}
    Dado $\Gamma \subseteq \mathcal{L}_{Prop}$ tal que $\Gamma = \{\alpha_1, \cdots, \alpha_n\}$ e $\beta \in \mathcal{L}_{Prop}$ tem-se que $\Gamma \vDash \beta$ se, e somente, se $\vDash \alpha_1 \Rightarrow ( \cdots ( \alpha_n \Rightarrow \beta))$.
\end{prop}

\begin{proof}
    A demonstração desse resultado pode ser verificada com $n$ aplicações de Corolário \ref{col:TeoremaDeducaoSemantico}.
\end{proof}

Um aspecto semântico importante do sistema axiomático $L$ é exposto pelos dois resultados a seguir.

\begin{theorem}\label{teo:AxiomasTautologiaEmL}
    Todos os axiomas do sistema axiomática $L$ são tautologias.
\end{theorem}

\begin{proof}
    Trivial, basta o leitor construir as tabelas verdades.
\end{proof}

\begin{corollary}\label{col:AxiomasTautologiaEmL}
    O conjunto de axiomas do sistema axiomática $L$ é satisfatível.
\end{corollary}

\begin{proof}
    Direto Teorema \ref{teo:AxiomasTautologiaEmL} e da definição de conjunto satisfaztível.
\end{proof}

Outro importante aspecto semântico entre as palavras de $\mathcal{L}_{Prop}$ é a noção de equivalência semântica definida formalmente a seguir.

\begin{definition}[Equivalência semântica]\label{def:EquivalenciaSemantica}
    Dado $\alpha, \beta \in \mathcal{L}_{Prop}$ é dito que $\alpha$ e $\beta$ são equivalente, denotado por $\alpha \equiv_\vDash \beta$, quando para toda interpretação $I_\rho$ tem-se que $I_\rho(\alpha) = I_\rho(\beta)$.
\end{definition}

\begin{rema}
    A Definição \ref{def:EquivalenciaSemantica} pode ser interpretada em termos de tabelas verdades da seguinte forma, duas palavras $\alpha$ e $\beta$ são equivalentes todas as linhas de suas tabelas verdades coincidem.
\end{rema}

\begin{exem}
    As palavras $\alpha \Rightarrow \beta$ e $\neg \alpha \lor \beta$ são semanticamente equivalentes como pode ser verificado na tabela a seguir.
    \begin{table*}[ht]
        \centering
        \scriptsize
        \begin{tabular}{|c|c|c|c|c|}
             \hline
             $P$ & $Q$ & $\neg P$ & $P \Rightarrow Q$ & $\neg P \lor Q$\\
             \hline
             0 & 0 & 1 & 1 & 1\\ \hline
             0 & 1 & 1 & 1 & 1\\ \hline
             1 & 0 & 0 & 0 & 0\\ \hline
             1 & 1 & 0 & 1 & 1\\ \hline
        \end{tabular}
    \end{table*}
\end{exem}

Em alguns textos (como \cite{BenjaV1}) é feita uma distinção entre lógica proposicional e a linguagem proposicional, isso acontece pois nesse tipo de texto é empregada a visão da lógica proposicional como sendo a ciência que estudo o raciocínio proposicional correto (ou verdadeiro), assim as palavras que não são tautologias não são diretamente interessantes no estudo da lógica proposicional nesta visão, entretanto, considerar o estudo da linguagem proposicional como um todo ou apenas de suas tautologia não há qualquer diferença\footnote{Isso ocorre pelo conjunto de tautologias e seu complemento serem ambos decidíveis. Assim a união dos dois, que é a própria linguagem proposicional, é também um conjunto decidível, e portanto, as propriedades válidas (meta-teoremas) checáveis para o conjunto das tautologia também são checáveis para a linguagem proposicional.} como apontado pelos professores Aho e Ullman em \cite{ullman1992}.

\section{Corretude e Completude}

Antes de introduzir os conceitos e resultados de completude e corretude é conveniente antes, apresentar algumas noções sobre o sistema sintático e o sistema semântico, apresentados nas seções anteriores. 

A ideia apresentada nas demonstrações do sistema sintático é o uso das regras de inferência para \textbf{derivar} palavras da linguagem proposicional, assim o sistema sintático, através da relação $\vdash$, de pode ser visto como um sistema de reescrita \cite{ayala2014}. Por outro lado, como discutido anteriormente o sistema semântico apresenta o mecanismo (a relação $\vDash$) para \textbf{validar} o significado das palavras da linguagem proposicional. Agora é momento em que o leitor perspicaz faz o questionamento: ``As relações $\vdash$ e $\vDash$ interagem de alguma forma?''. A resposta a esse questionamento é afirmativa e tais interações acontecem exatamente através da corretude e da completude como será mostrado nesta seção, a Figura \ref{fig:LogicaProposicional} a seguir ilustras as interações mencionadas neste parágrafo.

\begin{figure}[ht]
    \centering
    \begin{tikzpicture}[>=stealth, shorten >=1pt, node distance=3.0cm, on grid,
	auto, state/.append style={minimum size=2em}, thick]
    	\node[state, rectangle]        (A)               {$\mathcal{L}_{Prop}$};
    	\node[ ]                       (D)[below of=A]   {};
    	\node[state, rectangle]        (B)[left of=D]   {Derivabilidade ($\vdash$)};
    	\node[state, rectangle]        (C)[right of=D]   {Validade ($\vDash$)};
    	%\path[->] (A) +(-1,0) edge (A)
    	\path[->] 
    	(A) edge 			  				node [left] {Sintaxe} 				    (B)
    	(A) edge 			  				node [right] {Semântica} 				(C)
    	(B) edge [bend left]                node   {Corretude}                      (C)
    	(C) edge [bend left]                node   {Completude}                     (B);
    \end{tikzpicture}
    \caption{Relacionamento das estruturas que compõem a lógica proposicional.}
    \label{fig:LogicaProposicional}
\end{figure}

Este texto irá prosseguir primeiro mostrando que a lógica proposicional é correta, isto é, será demonstrado o teorema da corretude. De forma amigável tal teorema pode ser interpretado da seguinte forma, se existir uma prova de $\Gamma \vdash \alpha$, isto é, se for possível deduzir $\alpha$ a partir das premissas em $\Gamma$, isso significa que todo modelo de $\Gamma$ é também um modelo para o conjunto $\{\alpha\}$.

\begin{rema}\label{rema:LinguagemUsada}
    Como explicado devido a noção de abreviações o uso sistema axiomático $L$ e da linguagem implicativa $\mathcal{L}_{\Rightarrow}$ não diminui a generalidade da lógica proposicional, assim as demonstrações nessa seção irão considerar o sistema $L$ e $\mathcal{L}_{\Rightarrow}$ em vez da dedução natural e da linguagem $\mathcal{L}_{Prop}$. Para um leitor interessado na prova da corretude usado todo o sistema de dedução natural e a linguagem $\mathcal{L}_{Prop}$ é recomendado a leitura de \cite{joaoPavao2014}.
\end{rema}

\begin{lema}\label{lema:Corretude}
    Para qualquer $\alpha \in \mathcal{L}_{\Rightarrow}$ se $\vdash_L \alpha$, então $\vDash \alpha$.
\end{lema}

\begin{proof}
    A prova será realizada por indução sobre o tamanho\footnote{Lembre que o tamanho de uma demonstração em um sistema axiomático e o número de linhas da tabela que representa a prova.} $n$ das demonstrações no sistema $L$ com os seguintes passos de indução. 
    \begin{itemize}
        \item \textbf{Base da indução}: Se a demonstração de $\Gamma \vdash_L \alpha$ tem apenas uma linha, então $\alpha$ é um axioma, logo pelo Teorema \ref{teo:AxiomasTautologiaEmL} tem-se que $\alpha$ é uma tautologia, e portanto, $\vDash \alpha$.
        \item \textbf{Passo indutivo}: Existe uma prova com $n$ ou menos linhas de que $\vdash_L \alpha$ sendo que $\alpha$ é uma tautologia e:
        \begin{itemize}
            \item[(a)]  $\alpha$ é um axioma ou
            \item[(b)] $\alpha$ é obtido da aplicação de $E\Rightarrow$ nas linhas $i$ e $j$ tal que $i < j < n$.
        \end{itemize}
    \end{itemize}
    Assumindo a hipótese indutiva, suponha que exista uma prova de $n+1$ linha de que $\vdash_L \alpha$, assim se $\alpha$ é um axioma então pelo Teorema \ref{teo:AxiomasTautologiaEmL} tem-se que $\alpha$ é uma tautologia, e portanto, $\vDash \alpha$. De outra forma,  se $\alpha$ é obtido da aplicação de $E\Rightarrow$ nas linhas $i$ e $j$ com $i < j$, significa que existem  provas de $\vdash_L \alpha_0$ e $\vdash_L \alpha_0 \Rightarrow \alpha$ respectivamente com $i$ e $j$ linhas, assim pela hipótese indutiva tem-se que $\vDash \alpha_0$ e $\vDash \alpha_0 \Rightarrow \alpha$ de forma que $\alpha_0$ e $\alpha_0 \Rightarrow \alpha$ são tautologias, consequentemente para toda interpretação $I_\rho$ tem-se que $I_\rho(\alpha_0) = 1$ e $I_\rho(\alpha_0 \Rightarrow \alpha) = 1$, mas pela Definição \ref{def:interpretacat} pode-se concluir que $I_\rho(\alpha) = 1$, e portanto, $\alpha$ também é uma tautologia, o que completa a prova.
\end{proof}

\begin{theorem}[Teorema da corretude]
    Para quaisquer $\Gamma \subseteq \mathcal{L}_{\Rightarrow}$ e $\alpha \in \mathcal{L}_{\Rightarrow}$ se $\Gamma \vdash_L \alpha$, então $\Gamma \vDash \alpha$.
\end{theorem}

\begin{proof}
    Sem perda de generalidade pelas observações \ref{rema:EquivalenciaDeducaoNaturalAxiomatico} e \ref{rema:LinguagemUsada} junto com o Teorema \ref{teo:TeoremaDaCompacidade} pode-se assumir que $\Gamma = \{\alpha_1, \cdots, \alpha_n\}$ para algum $n \in \mathbb{N}$. Agora suponha que $\Gamma \vdash_L \alpha$, assim pelo Corolário \ref{col:DeducaoSintatica} e a Observação \ref{rema:EquivalenciaDeducaoNaturalAxiomatico} tem-se que $\vdash_L \alpha_1 \Rightarrow ( \cdots ( \alpha_n \Rightarrow \beta))$ e pelo Lema \ref{lema:Corretude} tem-se que $\vDash \alpha_1 \Rightarrow ( \cdots ( \alpha_n \Rightarrow \beta))$, mas disto e pela Propriedade \ref{prop:DeducaoSemantica} tem-se que  $\Gamma \vDash \alpha$, o que completa a prova.
\end{proof}

Para tratar da questão da completude, ou seja, para poder exibir um teorema da completude para a lógica proposicional, antes é necessário adicionar mais alguns conceitos para a sintaxe da linguagem.

\begin{definition}[Conjunto consistente]\label{def:ConjuntoConsistente}
    Um conjunto $\Gamma \subseteq (\mathcal{L}_{\Rightarrow} - \{\bot\})$ é consistente se, e somente se, para nenhum $\alpha \in (\mathcal{L}_{\Rightarrow} - \{\bot\})$ tem-se que $\Gamma \vdash_L \neg (\alpha \Rightarrow \alpha)$.
\end{definition}

\begin{definition}[Conjunto inconsistente]\label{def:ConjuntoInconsistente}
    Um conjunto $\Gamma \subseteq (\mathcal{L}_{\Rightarrow} - \{\bot\})$ é inconsistente se , e somente se, existe $\alpha \in (\mathcal{L}_{\Rightarrow} - \{\bot\})$ tal que $\Gamma \vdash_L \neg (\alpha \Rightarrow \alpha)$.
\end{definition}

\begin{rema}
    Note que as Definições \ref{def:ConjuntoConsistente} e \ref{def:ConjuntoInconsistente} podem ser escritas em relação a linguagem $\mathcal{L}_{Prop}$ substituindo $\neg (\alpha \Rightarrow \alpha)$ pela palavra $\alpha \land \neg \alpha$, isto pode ser facilmente verificado usando as abreviações apresentadas na Definição \ref{def:Abreviacoes}.
\end{rema}

\begin{definition}[Conjunto consistente]\label{def:ConjuntoMaxConsistente}
    Um conjunto $\Gamma \subseteq (\mathcal{L}_{\Rightarrow} - \{\bot\})$ é dito maximamente consistente se , e somente se, para todo $\alpha \in (\mathcal{L}_{\Rightarrow} - \{\bot\})$ as duas condições a seguir são satisfeitas:
    \begin{itemize}
        \item[(a)] $\Gamma$ é consistente;
        \item[(b)] $\alpha \in \Gamma$ e $\neg \alpha \notin \Gamma$ ou $\neg \alpha \in \Gamma$ e $\alpha \notin \Gamma$.
    \end{itemize}
\end{definition}

Como dito em \cite{joaoPavao2014} um conjunto maximamente consistente tem a propriedade de ser o maior possível sem ser inconsistente. 

\begin{theorem}\label{teo:ExtensaomaximamenteConsistente}
    Qualquer conjunto consistente $\Gamma$ pode ser entendido para um conjunto maximamente consistente $\Gamma_\infty$.
\end{theorem}

\begin{proof}
    Antes de qualquer coisa lembre-se que a linguagem $\mathcal{L}_{\Rightarrow}$ é também recursivamente enumerável (detalhes em \cite{ullman1992}), portanto, $(\mathcal{L}_{\Rightarrow} - \{\bot\})$ é também um conjunto recursivamente enumerável, assim as palavras de $(\mathcal{L}_{\Rightarrow} - \{\bot\})$ podem ser distribuídas em uma sequencia ordenada $[\alpha_1, \alpha_2, \cdots]$, dito isto, suponha que $\Gamma$ é um conjunto consistente e defina indutivamente os seguintes conjuntos:
    \begin{eqnarray*}
        \Gamma_0 & = & \Gamma\\
        \Gamma_1 & = & \left\{\begin{array}{ll}	\Gamma_0 \cup \{\alpha_1\}, & \hbox{se } \Gamma_1 \cup \{\beta_1\} \hbox{ é consistente}\\ \Gamma_0 \cup \{\neg \alpha_1\},  & \hbox{senão.}\end{array}\right.\\
        \Gamma_{n+1} & = & \left\{\begin{array}{ll}	\Gamma_n \cup \{\alpha_{n+1}\}, & \hbox{se } \Gamma_1 \cup \{\beta_{n+1}\} \hbox{ é consistente}\\ \Gamma_n \cup \{\neg \alpha_{n+1}\},  & \hbox{senão.}\end{array}\right.
    \end{eqnarray*}
    Obviamente por esta construção, para qualquer que seja $i \in \mathbb{N}$ tem-se que $\Gamma_i$ será consistente. Agora considere o conjunto:
    $$\Gamma_\infty = \bigcup_{i \in \mathbb{N}} \Gamma_i$$
    suponha por absurdo que $\Gamma_\infty$ não é um conjunto consistente,  assim existe um $\alpha$ tal que   $\Gamma_\infty \vdash_L \neg (\alpha \Rightarrow \alpha)$, assuma que $\neg (\alpha \Rightarrow \alpha)$ corresponde a palavra $\alpha_i$ da sequência ordenada $[\alpha_1, \alpha_2, \cdots]$, portanto, é obvio que $\Gamma_i \vdash_L \neg (\alpha \Rightarrow \alpha)$, o que é uma contradição, uma vez que, $\Gamma_i$ é consistente, consequentemente, $\Gamma_\infty$ é consistente. De forma trivial, pela construção de $\Gamma_\infty$ tem-se que este atende a condição (b) da Definição \ref{def:ConjuntoMaxConsistente}. Desde que, $\Gamma_\infty$ é consistente e atende a condição (b), tem-se que $\Gamma_\infty$ é um conjunto maximamente consistente, o que conclui a prova.
\end{proof}

\begin{theorem}\label{teo:ModeloParaMaximaneteConsistente}
    Qualquer conjunto maximamente consistente $\Gamma$ tem um modelo.
\end{theorem}

\begin{proof}
    Ver a demonstração em \cite{joaoPavao2014}.
\end{proof}

\begin{lema}\label{lema:Completude}
    Para $\alpha \in \mathcal{L}_{\Rightarrow}$ se $\vDash \alpha$, então $\vdash_L \alpha$.
\end{lema}

\begin{proof}
    Dado o conjunto $\Gamma$ de todos os teoremas do sistema axiomático $L$, isto é, $\Gamma = Th(\emptyset)$. Suponha por absurdo que existe uma palavra $\alpha$ tal que $\alpha$ seja uma tautologia (ou seja, $\vDash \alpha$), mas que $\alpha$ não seja um teorema de $L$. Como $\alpha \notin \Gamma$ considere o conjunto $\Gamma \cup \{\neg \alpha\}$, se tal conjunto for inconsistente então $\alpha \in \Gamma$ o que é um absurdo pois contradiz a hipótese de $\alpha$ não ser um teorema. Por outro lado, se $\Gamma \cup \{\neg \alpha\}$ é consistente pelo Teorema \ref{teo:ExtensaomaximamenteConsistente} existe um conjunto maximamente consistente $\Gamma_\infty$ obtido a partir de $\Gamma \cup \{\neg \alpha\}$, agora pelo Teorema \ref{teo:ModeloParaMaximaneteConsistente} existe um modelo $I_\rho$ para $\Gamma_\infty$, e assim $I_\rho(\neg \alpha) = 1$, o que implica que $I_\rho(\alpha) = 0$, o que é uma contradição, e portanto, $\alpha$ é um teorema de $L$, ou seja, $\vdash_L \alpha$.
\end{proof}

\begin{theorem}[Teorema da completude]\label{teo:Completude}
    Para quaisquer $\alpha_1, \cdots, \alpha_n, \alpha \in \mathcal{L}_{\Rightarrow}$ se $\{\alpha_1, \cdots, \alpha_n\} \vDash \alpha$, então $\{\alpha_1, \cdots, \alpha_n\} \vdash_L \alpha$.
\end{theorem}

\begin{proof}
    Suponha que $\{\alpha_1, \cdots, \alpha_n\} \vDash \alpha$ assim pela Proposição \ref{prop:DeducaoSemantica} tem-se que $\vDash \alpha_1 \Rightarrow ( \cdots ( \alpha_n \Rightarrow \alpha))$ dessa forma pela Lema \ref{lema:Completude} tem-se que $\vdash_L \alpha_1 \Rightarrow ( \cdots ( \alpha_n \Rightarrow \alpha))$, agora aplicando $n$ vezes o Teorema \ref{teo:TeoremaDeducaoSintatico2}  tem-se que $\{\alpha_1, \cdots, \alpha_n\} \vdash_L \alpha$, o que completa a prova.
\end{proof}

Como explicado em \cite{joaoPavao2014} a completude apresentada pelo teorema acima é a chamada completude fraca pois o conjunto de premissas $\Gamma$ é um conjunto finito, entretanto, este resultado pode ser estendido para um conjunto de premissas infinito \cite{joaoPavao2014}.

\section{Resolução}

A resolução foi introduzida pelo filósofo, matemático e cientista da computação John Alan Robinson (1930--2016) em seu trabalho \cite{robinson1965} e depois generalizado pelo próprio Robinson em \cite{robinson1983}. A resolução corresponde a uma abordagem contendo apenas uma única regra de inferência, a saber o princípio da resolução. Tal abordagem é uma alternativa aos sistemas dedutivo (dedução natural) e axiomático, ou seja, é uma abordagem sintática.

Como mencionado em \cite{joaoPavao2014} o trabalho inicial sobre resolução feito por Robinson também pode ser visto como marco inicial do que hoje é chamado de demonstração automática de teoremas\footnote{Em alguns texto também é usado o termo raciocínio automático \cite{robinson2001}.} e em especial é responsável pelo surgimento da linguagem prolog \cite{ayala2014}.

A resolução necessita das noções de cláusula e forma normal conjuntiva, assim é conveniente introduzir estes antes do princípio da resolução em si.

\begin{definition}[Literais]\label{def:Literais}
    Para todo $\alpha \in \Sigma_s$ tem-se que $\alpha$ e $\neg \alpha$ são chamados respectivamente de literal positivo e literal negativo.
\end{definition}

\begin{exem}
    São exemplos de literais positivos $P, Q_1, R_2$ e são literais negativos $\neg P, \neg R_3$ e $\neg Q_1$.
\end{exem}

\begin{definition}[cláusula]\label{def:cláusula}
    Uma cláusula $C$ é um literal ou a disjunção de literais.
\end{definition}

\begin{exem}
    são exemplos de cláusulas $P, \neg P \lor \neg Q, \neg P \lor P, P \lor Q$ e $neg Q$.
\end{exem}

\begin{definition}[cláusula unitária]\label{def:cláusulaUnitaria}
    Uma cláusula $\phi$ é dita unitária se ela é um literal. 
\end{definition}

\begin{definition}[Forma normal conjuntiva]\label{def:FNC}
    Uma palavra $\alpha \in \mathcal{L}_{Prop}$ é dita está na forma normal conjuntiva, ou simplesmente FNC, se $\alpha$ é da forma $\beta_1 \land \cdots \land \beta_n$ tal que para todo $1 \leq i \leq n$ tem-se que $\beta_i$ é uma cláusula.
\end{definition}

\begin{exem}
    A palavra $P \land (\neg P \lor Q) \land (\neg Q \lor Q)$ está na FNC. Já a palavra $P \Rightarrow (Q \lor \neg Q)$ não está na FNC.
\end{exem}

Como explicado em \cite{joaoPavao2014} uma forma conveniente de representar cláusulas e palavras na FNC é usando conjuntos, para o caso das cláusulas as mesmas corresponde a um conjunto de literais, e para as palavras em FNC as mesmas podem ser vistas como um conjunto de cláusulas.

\begin{exem}
    Dado a palavra na FNC $P \land (\neg P \lor Q) \land \neg R$ a mesma pode ser representada pelo seguinte conjunto de cláusulas:
    $$\{P, \neg P \lor Q, \neg R \}$$
    Agora as cláusulas corresponde aos seguintes conjuntos:
    \begin{itemize}
        \item $\{P\}$
        \item $\{\neg P, Q\}$
        \item $\{\neg R\}$
    \end{itemize}
    Assim a palavra $P \land (\neg P \lor Q) \land \neg R$ pode ser representada pelo seguinte conjunto:
    $$\{\{P\}, \{\neg P, Q\}, \{\neg R\}\}$$
\end{exem}

Agora será apresentado um mecanismo, na forma de algoritmo, que é capaz de converter uma palavra qualquer $\alpha \in \mathcal{L}_{Prop}$ para uma família finita de conjuntos de literais. Para a descrição deste método que será apresentada a seguir, para qualquer $\alpha, \beta \in \mathcal{L}_{Prop}$ a expressão $\alpha \boxdot \beta$ deverá ser interpretada como um relação de reescrita, em que $\alpha$ é reescrito como $\beta$, vale ressaltar que tal relação é bi-lateral, ou seja, também pode ser aplicada no sentido inverso, isto é, $\beta$ pode ser reescrito como $\alpha$. Dito isso, agora será apresentado as regras repensáveis pela conversão para a FNC, muitas dessas regras como dito em \cite{joaoPavao2014} na verdade são abreviações.

\begin{definition}[Regra da conversão da implicação]\label{Def:ConversaoImplicacao}
    Para qualquer $\alpha, \beta \in \mathcal{L}_{Prop}$ tem-se que $\alpha \Rightarrow \beta \boxdot \neg \alpha \lor \beta$.
\end{definition}

\begin{exem}\label{exe:ConversaoFNC1}
    Dado a palavra $\neg P \Rightarrow (Q \Rightarrow \neg (\neg Q \land R))$ tem-se que 
    $$\neg P \Rightarrow (Q \Rightarrow \neg (\neg Q \land R)) \boxdot \neg \neg P \lor \neg (\neg Q \lor (\neg Q \land R))$$
\end{exem}

\begin{definition}[Redução do escopo da negação]
    A redução do escopo da negação, ou simplesmente RN, é uma regra que é apresenta as três formas, sendo elas:
    \begin{itemize}
        \item[ ] (RN$_1$) Para todo $\alpha \in \mathcal{L}_{Prop}$ tem-se que $\neg \neg \alpha \boxdot \alpha$.
        \item[ ] (RN$_2$) Para todo $\alpha, \beta \in \mathcal{L}_{Prop}$ tem-se que $\neg(\alpha \land \beta) \boxdot \neg \alpha \lor \neg \beta$.
        \item[ ] (RN$_3$) Para todo $\alpha, \beta \in \mathcal{L}_{Prop}$ tem-se que $\neg(\alpha \lor \beta) \boxdot \neg \alpha \land \neg \beta$.
    \end{itemize}
\end{definition}

\begin{exem}
    Dado a palavra $\neg \neg P \lor (\neg Q \lor \neg (\neg Q \land R))$ obtida ao final do Exemplo \ref{exe:ConversaoFNC1} tem-se que
    \begin{eqnarray*}
        \neg \neg P \lor (\neg Q \lor \neg (\neg Q \land R)) & \boxdot & P \lor (\neg Q \lor \neg (\neg Q \land R))\\
        & \boxdot & P \lor (\neg Q \lor (\neg\neg Q \lor \neg R))\\
        & \boxdot & P \lor (\neg Q \lor (Q \lor \neg R))
    \end{eqnarray*}
\end{exem}

\begin{rema}
    Na literatura as formas 2 e 3 da regra de redução do escopo da negação costumam ser chamadas de primeiras lei De Morgan em homenagem ao matemático e lógico britânico Augustus De Morgan (1806--1871).
\end{rema}


\begin{definition}[Regra da distribuição]\label{def:RegraDaDistribuicao}
    A regra da distribuição é dado para todo $\alpha, \beta, \gamma \in \mathcal{L}_{Prop}$ como sendo, $\alpha \lor (\beta \land \gamma) \boxdot (\alpha \lor \beta) \land (\alpha \lor \gamma)$.
\end{definition}

\begin{exem}
    Dado a palavra $\neg P \lor (R_1 \land \neg R_2)$ aplicando a regra da distribuição é obtida a palavra $(\neg P \lor R_1) \land (\neg P \lor \neg R_2)$.
\end{exem}

\begin{definition}[Remoção da conjunção]\label{def:RegraDeRemoverE}
    Dada a palavra $\alpha \in \mathcal{L}_{Prop}$ na forma normal conjuntiva, isto é, $\alpha = \beta_1 \land \cdots \land \beta_2$ tal que para todo $1 \leq i \leq n$ tem-se que $\beta_i$ é uma cláusula. A regra de remoção da conjunção consiste em transformar $\alpha$ no conjunto $C_\alpha = \{\beta_1, \cdots, \beta_n\}$. 
\end{definition}

\begin{exem}
    Dado a palavra $\alpha = (P \lor Q) \land (\neg P \lor R) \land \neg Q$ aplicando a regra de remoção da conjunção é obtido o conjunto $C_\alpha = \{P \lor Q, \neg P \lor R, \neg Q\}$.
\end{exem}

\begin{definition}[Remoção da disjunção]\label{def:RegraDeRemoverOu}
    Dado um conjunto de cláusulas $C_\alpha = \{\beta_1, \cdots, \beta_n\}$, a regra de remoção da disjunção consiste em obter a família finita dos conjuntos de literais das cláusulas em $C_\alpha$, isto é, a regra de remoção da disjunção consiste em construir o conjunto:
    $$FCL_\alpha = \{Li_\beta \mid \beta \in C_\alpha\}$$
    onde $Li_\beta$ corresponde ao conjunto de literais na cláusula $\beta$.
\end{definition}

\begin{exem}
    Dado o conjunto de cláusulas  $\alpha = \{P \lor Q, \neg P \lor R, \neg Q\}$ aplicando a regra de remoção da disjunção é construído o conjunto $FCL_\alpha = \{\{P, Q\}, \{\neg P, R\}, \{\neg Q\}\}$.
\end{exem}

Todas estas regras podem ser combinadas em um único algoritmo de forma que para qualquer que seja a palavra $\alpha \in \mathcal{L}_{Prop}$ sempre é possível obter uma família finita de conjuntos de literais de $\alpha$ como pode ser visto a seguir.

\begin{algorithm}[h]
    \Entrada{$\alpha \in \mathcal{L}_{Prop}$}
    \Saida{Uma família finita de conjuntos de literais $FCL_\alpha$.}
    \Inicio{
        \Repita{obter a FNC de $\alpha$}{
            Aplique a regra conversão da implicação\\
            Aplique a regra de redução do escopo da negação\\
            Aplique a regra da distribuição
        }
        Obtenha $C_\alpha$\\
        Obtenha $FCL_\alpha$\\
        \Retorna{$FCL_\alpha$}
    }
    \caption{Algoritmo para converter uma palavra para uma família finita de conjuntos de literais.}
    \label{alg:PreResolucao}
\end{algorithm}

A família finita $FCL_\alpha$ obtida pelo Algoritmo \ref{alg:PreResolucao} será chamada de componente de entrada para a utilização da resolução, pois como explicado em \cite{joaoPavao2014, ayala2014} tal família é exatamente a entrada para o método de resolução.

\begin{definition}
    Seja $\Gamma \subseteq \mathcal{L}_{Prop}$ com $\Gamma = \{\alpha_1, \cdots, \alpha_n\}$ o conjunto $FCL$ de $\Gamma$, denotado por $FCL(\Gamma)$, corresponde a uma família de conjuntos de literais na forma,
    $$FCL(\Gamma) = \bigcup_{\alpha \in \Gamma} FCL_\alpha$$
\end{definition}

\begin{exem}
    Dado o conjunto $\Gamma = \{(\neg\neg P \lor Q) \land (S \lor R), \neg(S \lor \neg  Q)\}$ fazendo $\alpha_1 = (\neg\neg P \lor Q) \land (S \lor R)$ e $\alpha_2 = \neg(S \lor \neg  Q)$ tem-se,
    $$FCL_{\alpha_1} = \{\{P, Q\}, \{S, R\}\}$$
    $$FCL_{\alpha_2} = \{\{\neg  S\}, \{Q\}\}$$
    portanto, 
    $$FCL(\Gamma) = \{\{P, Q\}, \{S, R\}, \{\neg S\}, \{Q\}\}$$
\end{exem}


Agora que foram apresentadas todas as definições básicas necessárias para o estudo da resolução pode-se seguir com este texto apresentado a princípio da resolução.

\begin{definition}[Princípio da resolução]
    Dado $Li_1$  e $Li_2$ dois conjuntos finitos e não vazios de literais e seja $\alpha \in \Sigma_s$, se $\alpha \in Li_1$ e $\neg \alpha \in Li_2$, então é inferido o conjunto de literais $(Li_1 - \{\alpha\}) \cup (Li_2 - \{\neg \alpha\})$.
\end{definition}


É comum encontrar na literatura (ver \cite{joaoPavao2014, BenjaV1}) a representação do princípio da resolução como sendo uma árvore, onde a raiz da árvore contém exatamente o conjunto de literais obtidos pela aplicação do princípio da resolução, tal conjunto recebe o nome de \textbf{resolvente}, um exemplo dessa representação pode ser visto a seguir.

\begin{figure}[ht]
    \centering
    \begin{forest}
      for tree={
        grow'=90,
        parent anchor=north,
        math content,
        before typesetting nodes={
          if level=0{}{
            if content={}{
              shape=coordinate
            }{
              content/.wrap value={\{#1\}},
            },
          },
        }
      }
      [(Li_1 - \{\alpha\}) \cup (Li_2 \cup \{\neg \alpha\})
        [
          {Li_1}
        ]
        [
          {Li_2}
        ]
      ]
    \end{forest}
    \caption{Representação visual do princípio da resolução.}
    \label{fig:PrincipioDaResolucao}
\end{figure}

\begin{rema}
    A aplicação do princípio da resolução entre dois conjuntos de literais $Li_1$ e $Li_2$ é representado simplesmente por $RES(Li_1, Li_2)$.
\end{rema}

\begin{exem}
    Dado $Li_1 = \{P, Q, \neg P\}$ e $Li_2 = \{\neg Q, R\}$ tem-se que $RES(Li_1, Li_2) = \{P, \neg P, R\}$.
\end{exem}

\begin{definition}[Prova usando resolução]\label{def:ProvaResolucao}
    Uma prova por resolução para um conjunto de literais $\alpha$ a partir de uma família de conjuntos de literais $FCL(\Gamma)$ é uma sequência de conjuntos de literais $\alpha_1, \cdots, \alpha_n$ tal que:
    \begin{enumerate}
        \item $\alpha_n = \alpha$.
        \item Para todo $1 \leq i \leq n$ tem-se que: 
        \begin{itemize}
            \item $\alpha_i \in FCL(\Gamma)$ ou 
            \item $\alpha_i= RES(\beta_1, \beta_2)$ com $\beta_1, \beta_2 \in \{\alpha_1, \cdots, \alpha_{i-1}\}$.
        \end{itemize}
    \end{enumerate}
\end{definition}


Alguns autores (ver \cite{BenjaV1}) utilizam uma notação de árvores para representar provas por resolução. Neste texto, entretanto, a representação (ou escrita) de uma prova utilizando resolução será feita da mesma forma que as provas no sistema axiomático, isto é, uma prova usando resolução irá consistir de uma tabela com três colunas onde a primeira coluna diz respeito ao número da linha, a segunda coluna é um conjunto de literais e a terceira e última coluna informa se o conjunto de literais é uma premissa ou se foi obtido através da aplicação do princípio da resolução em duas linhas anteriores da tabela.

\begin{exem}
    Seja $FCL(\Gamma) = \{\{\neg P, Q\},\{\neg Q, R\}, \{\neg R, S\}, \{P\} \}$ uma prova de $\{S\}$ a partir de $FCL(\Gamma)$ é dada por:
    \begin{table}[ht]
        \centering
        \begin{tabular}{|c|c|c|}
            \hline
            1 & $\{\neg P, Q\}$ & Premissa \\ \hline
            2 & $\{\neg Q, R\}$ & Premissa \\ \hline
            3 & $\{\neg R, S\}$ & Premissa \\ \hline
            4 & $\{P\}$ & Premissa \\ \hline
            5 & $\{Q\}$ & RES(1,4) \\ \hline
            6 & $\{R\}$ & RES(2,5) \\ \hline
            7 & $\{S\}$ & RES(3,6) \\ \hline
        \end{tabular}
    \end{table}
\end{exem}


\begin{definition}[Consequência sintática por resolução]
    Seja $\Gamma \subseteq \mathcal{L}_{Prop}$ e $\alpha \in \mathcal{L}_{Prop}$. Sempre que existir uma prova por resolução de $Li_{\alpha}$ a partir de $FCL(\Gamma)$, será dito que $\alpha$ é uma consequência sintática por resolução de $\Gamma$, e isto é denotado por $\Gamma \vdash_{RES} \alpha$.
\end{definition}

\begin{theorem}[Teorema da dedução para resolução]\label{teo:DeducaoResolucao}
    Se $\Gamma \cup \{\alpha\} \vdash_{RES} \beta$, então $\Gamma \vdash_{RES} \neg \alpha \lor \beta$.
\end{theorem}

\begin{proof}
    Fica como exercício ao leitor.
\end{proof}

É comum como descrito em \cite{joaoPavao2014}, adotar a resolução para provas por redução ao absurdo , nestes casos tais provas recebem o nome de refutação por resolução. Em tal tipo de prova a negação da conclusão (objetivo da prova) deve ser adicionada as premissas, e a partir deste novo conjunto de premissas deve-se chegar ao absurdo (conjunto vazio) usando a prova pro resolução. Ou seja, uma prova de refutação por resolução da palavra $\alpha$ a partir de um conjunto $\Gamma$, é uma prova de $\Gamma \cup \{\neg \alpha\} \vdash_{RES} \emptyset$, isto é formalizado a seguir.

\begin{definition}[Refutação por resolução]
    Seja $\Gamma \subseteq \mathcal{L}_{Prop}$ e $\alpha \in \mathcal{L}_{Prop}$. Uma prova por refutação de $Li_\alpha$ a partir de $FCL(\Gamma)$ é uma prova da forma $\Gamma \cup \{\neg \alpha\} \vdash_{RES} \emptyset$.
\end{definition}

\begin{corollary}\label{prop:DeducaoResolução}
    Se $\Gamma \cup \{\neg \alpha\} \vdash_{RES} \emptyset$, então $\Gamma \vdash_{RES} \alpha$.
\end{corollary}

\begin{proof}
    Direto pelo Teorema \ref{teo:DeducaoResolucao}.
\end{proof}

\begin{exem}
    A demonstração de que a palavra $(\neg P \land \neg Q) \Rightarrow \neg (P \lor Q)$ é um teorema usando refutação por resolução se dá da seguinte forma, primeiro a negação da conclusão deve ser adicionada ao conjunto de premissas, isto é, $\Gamma = \{\neg((\neg P \land \neg Q) \Rightarrow \neg (P \lor Q))\}$, agora deve-se provar que $\Gamma \vdash_{RES} \emptyset$, mas para isso é necessário calcular $FCL(\Gamma)$, para resolver esta questão é aplicado o Algoritmo \ref{alg:PreResolucao} é obtido que $FCL(\Gamma) = \{\{\neg P\}, \{\neg Q\}, \{P. Q\}\}$, agora tem-se a seguinte prova, 
    \begin{table}[ht]
        \centering
        \begin{tabular}{|c|c|c|}
            \hline
            1 & $\{\neg P\}$ & Premissa \\ \hline
            2 & $\{\neg Q\}$ & Premissa \\ \hline
            3 & $\{P. Q\}$ & Premissa \\ \hline
            4 & $\{P\}$ & RES(2,3) \\ \hline
            5 & $\emptyset$ & RES(1,4) \\ \hline
        \end{tabular}
    \end{table}
    
    isto é, $\{\neg((\neg P \land \neg Q) \Rightarrow \neg (P \lor Q))\} \vdash_{RES} \emptyset$, e portanto, pela Proposição \ref{prop:DeducaoResolução} tem-se que $\vdash_{RES} (\neg P \land \neg Q) \Rightarrow \neg (P \lor Q)$.
\end{exem}

Como apresentado em \cite{joaoPavao2014}, existem diversos procedimentos de se escolher quais componentes usar para aplicação da resolução, estes procedimentos recebem o nome de \textbf{estratégias de resolução}. Entre tais estratégias estão: a saturação por níveis\footnote{Usa a visão da resolução como uma árvore, e assim escolhe os componentes usado na resolução usando o algoritmo de busca em largura \cite{jaime1994}.}, eliminação de cláusulas, seleção de cláusulas entre outras. Além de detalhes das estratégias de resolução em \cite{joaoPavao2014} também são apresentados as provas para corretude e completude da lógica proposicional usando resolução. 

\section{Diagrama de Blocos}

Escrever futuramente!

\section{Questionário}

\begin{exercise}\label{exerc:LPro1}
    Usando o sistema dedutivo da lógica proposicional demostre as seguintes relações de consequência.
\end{exercise}
\begin{enumerate}
    \item $\vdash P \Rightarrow (Q \Rightarrow P)$
    \item $\{P \Rightarrow Q, Q \Rightarrow R\} \vdash P \Rightarrow R$
    \item $\{P \Rightarrow (Q \Rightarrow R), P \Rightarrow Q\} \vdash P \Rightarrow R$
    \item $\{(P \Rightarrow Q) \Rightarrow (P \Rightarrow R)\} \vdash P \Rightarrow (Q \Rightarrow R)$
    \item $\{P \Rightarrow (P \Rightarrow Q)\} \vdash P \Rightarrow Q$
    \item $\{P \Rightarrow (Q \Rightarrow R)\} \vdash Q \Rightarrow (P \Rightarrow R)$
    \item $\{P \Rightarrow (Q \Rightarrow (R \Rightarrow S))\} \vdash R \Rightarrow (Q \Rightarrow (P \Rightarrow S))$
    \item $\{(P \Rightarrow Q) \Rightarrow R\} \vdash P \Rightarrow (Q \Rightarrow R)$
    %\item $\vdash (P  \Rightarrow  Q)  \Rightarrow  P  \Rightarrow  (Q  \Rightarrow  R)  \Rightarrow  R$
    %\item $\vdash (P  \Rightarrow  Q  \Rightarrow  R)  \Rightarrow  P  \Rightarrow  (P  \Rightarrow  Q)  \Rightarrow  R$
    %\item $\vdash ((P  \Rightarrow  Q)  \Rightarrow  Q  \Rightarrow  R)  \Rightarrow  (P  \Rightarrow  Q)  \Rightarrow  P  \Rightarrow  R$
    %\item $\vdash (P  \Rightarrow  Q)  \Rightarrow  P  \Rightarrow  ((P  \Rightarrow  Q)  \Rightarrow  R  \Rightarrow  Q  \Rightarrow  S)  \Rightarrow  ((P  \Rightarrow  Q)  \Rightarrow  R)  \Rightarrow  S$
    \item $\{P \land  Q\} \vdash Q \land  P$
    \item $\{P \land  (Q \land  R)\} \vdash (P \land  Q) \land  R$
    \item $\{P \Rightarrow R\} \vdash (P \land  Q) \Rightarrow R$
    \item $\{P \Rightarrow Q\} \vdash (R \land  P) \Rightarrow (Q \land  R)$
    \item $\{P \Rightarrow (Q \Rightarrow R)\} \vdash (P \land  Q) \Rightarrow R$
    \item $\{(P \land  Q) \Rightarrow R\} \vdash P \Rightarrow (Q \Rightarrow R)$
    \item $\{(P \Rightarrow Q) \Rightarrow R\} \vdash (P \land  Q) \Rightarrow R$
    \item $\{P \land  (Q \Rightarrow R)\} \vdash (P \Rightarrow Q) \Rightarrow R$
    \item $\{(P \Rightarrow Q) \land  (P \Rightarrow R)\} \vdash P \Rightarrow (Q \land  R)$
    \item $\{P \Rightarrow (Q \land  R)\} \vdash (P \Rightarrow Q) \land  (P \Rightarrow R)$
    \item $\{P \lor P\} \vdash P$
    \item $\{P \lor Q\} \vdash Q \lor P$
    \item $\{P \lor (Q \lor R)\} \vdash (P \lor Q) \lor R$
    \item $\{(Q \lor (P \lor R)) \lor P\} \vdash R \lor (P \lor Q)$
    \item $\{P \Rightarrow Q\} \vdash P \Rightarrow (Q \lor R)$
    \item $\{(P \lor Q) \Rightarrow R\} \vdash P \Rightarrow R$
    \item $\{P \lor Q, P \Rightarrow R, Q \Rightarrow S\} \vdash R \lor S$
    \item $\{P \Rightarrow R, Q \Rightarrow R\} \vdash (P \lor Q) \Rightarrow R$
    \item $\{P \Rightarrow Q\} \vdash (R \lor P) \Rightarrow (Q \lor R)$
    \item $\{P \land (Q \lor R)\} \vdash (P \land Q) \lor (P \land R)$
    \item $\{(P \land Q) \lor (P \land R)\} \vdash P \land (Q \lor R)$
    \item $\{P \lor (Q \land R)\} \vdash (P \lor Q) \land (P \lor R)$
    \item $\{(P \lor Q) \land (P \lor R)\} \vdash P \lor (Q \land R)$
    \item $\{P \land Q\} \vdash P \lor Q$
    \item $\{P \land Q\} \vdash P \Rightarrow Q$
    \item $\{P\} \vdash P \land (P \lor Q)$
    \item $\{P \lor (P \land Q)\} \vdash P$
    %\item $\vdash P \lor \top$
    \item $\{P \land\bot \} \vdash Q$
    \item $\{\neg P\} \vdash P \Rightarrow  \bot$
    \item $\{P \Rightarrow  \bot\} \vdash \neg P$
    \item $\{P\} \vdash \neg \neg P$
    \item $\{\neg \neg \neg P\} \vdash \neg P$
    \item $\{\neg \neg P\} \vdash P$
    \item $\{P, \neg P\} \vdash Q$
    \item $\{P \Rightarrow Q, P \Rightarrow \neg Q\} \vdash \neg P$
    \item $\{\neg \neg P \Rightarrow Q, \neg \neg P \Rightarrow \neg Q\} \vdash \neg P$
    \item $\{\neg P \Rightarrow Q, \neg P \Rightarrow \neg Q\} \vdash P$
    \item $\vdash P \lor \neg P$
    \item $\{P \Rightarrow Q, \neg P \Rightarrow Q\} \vdash Q$
    \item $\vdash (P \Rightarrow \neg P) \Rightarrow \neg P$
    \item $\vdash (\neg P \Rightarrow P) \Rightarrow P$
    \item $\{\neg P \Rightarrow \neg Q, Q\} \vdash P$
    \item $\{\neg (P \land Q), P\} \vdash \neg Q$
    \item $\{\neg (P \land Q), Q\} \vdash \neg P$
    \item $\{\neg (P \land \neg Q), P\} \vdash Q$
    \item $\{\neg (\neg P \land Q), Q\} \vdash P$
    \item $\{\neg (P \land (Q \land R)), Q\} \vdash \neg (P \land R)$
    \item $\{\neg (P \land T \land Q), \neg (R \land \neg T \land S)\} \vdash \neg (P \land Q \land R \land S)$
    \item $\{P \lor Q, \neg P\} \vdash Q$
    \item $\{\neg P \lor Q, P\} \vdash Q$
    \item $\{P \lor Q, \neg Q\} \vdash P$
    \item $\{P \lor \neg Q, Q\} \vdash P$
    \item $\{P \lor (T \lor Q), R \lor (\neg T \lor S)\} \vdash (P \lor Q) \lor (R \lor S)$
    \item $\{\neg P\} \vdash \neg (P \land Q)$
    \item $\{\neg Q\} \vdash \neg (P \land Q)$
    \item $\{P\} \vdash \neg (\neg P \land Q)$
    \item $\{Q\} \vdash \neg (P \land \neg Q)$
    \item $\{\neg (P \lor Q)\} \vdash \neg P$
    \item $\{\neg (P \lor Q)\} \vdash \neg Q$
    \item $\{\neg (\neg P \lor Q)\} \vdash P$
    \item $\{\neg (P \lor \neg Q)\} \vdash Q$
    \item $\{P \Rightarrow Q\} \vdash \neg P \lor Q$
    \item $\{\neg P \Rightarrow Q\} \vdash P \lor Q$
    \item $\{P \Rightarrow Q\} \vdash \neg (P \land \neg Q)$
    \item $\{P \Rightarrow \neg Q\} \vdash \neg (P \land Q)$
    \item $\{P, \neg Q\} \vdash \neg (P \Rightarrow Q)$
    \item $\{\neg (P \Rightarrow Q)\} \vdash P$
    \item $\{\neg (P \Rightarrow Q)\} \vdash \neg Q$
    \item $\{\neg (P \Rightarrow \neg Q)\} \vdash Q$
    \item $\{\neg P \land \neg Q\} \vdash \neg (P \lor Q)$
    \item $\{\neg P \land \neg Q, P \lor Q\} \vdash \neg (\neg P \land \neg Q)$
    \item $\{\neg (\neg P \land \neg Q)\} \vdash (P \lor Q)$
    \item $\{P \land Q\} \vdash \neg (\neg P \lor \neg Q)$
    \item $\{\neg (\neg P \lor \neg Q)\} \vdash (P \land Q)$
    \item $\{\neg P \lor \neg Q\} \vdash \neg (P \land Q)$
    \item $\{\neg (P \land Q)\} \vdash \neg P \lor \neg Q$
    \item $\{\neg (\neg P \land (Q \land \neg R)), Q\} \vdash P \lor R$
\end{enumerate}

\begin{exercise}\label{exerc:LPro2}
    Considerando a linguagem proposicional enriquecida pelo símbolo de bi-implicação ($\Leftrightarrow$) demonstre usando o sistema dedutivo as seguintes relações de consequência.
\end{exercise}

\begin{enumerate}
    \item $\vdash P \Leftrightarrow P$
    \item $\{P \Leftrightarrow Q\} \vdash Q \Leftrightarrow P$
    \item $\{P \Leftrightarrow Q, Q \Leftrightarrow R\} \vdash P \Leftrightarrow R$
    \item $\{P, Q\}\vdash P \Leftrightarrow Q$
    \item $\{\neg P, \neg Q\} \vdash P \Leftrightarrow Q$
    \item $\{P, \neg Q\} \vdash \neg (P \Leftrightarrow Q)$
    \item $\{\neg P, Q\} \vdash \neg (P \Leftrightarrow Q)$
    \item $\{P \Leftrightarrow Q, \neg P\} \vdash \neg Q$
    \item $\{P \Leftrightarrow Q, \neg Q\} \vdash \neg P$
    \item $\{\neg P \Leftrightarrow Q, P\} \vdash \neg Q$
    \item $\{P \Leftrightarrow \neg Q, Q\} \vdash \neg P$
    \item $\{\neg (P \Leftrightarrow Q), P\} \vdash \neg Q$
    \item $\{\neg (P \Leftrightarrow Q), Q\} \vdash \neg P$
    \item $\{\neg (P \Leftrightarrow Q), \neg P\} \vdash Q$
    \item $\{\neg (P \Leftrightarrow Q), \neg Q\} \vdash P$
    \item $\{P \Leftrightarrow Q, P \Leftrightarrow \neg Q\} \vdash \bot$
    \item $\vdash \neg (P \Leftrightarrow Q) \Leftrightarrow \neg (Q \Leftrightarrow P)$
    \item $\vdash \neg (P \Leftrightarrow Q) \Leftrightarrow (\neg P \Leftrightarrow Q)$
    \item $\vdash \neg (P \Leftrightarrow Q) \Leftrightarrow (P \Leftrightarrow \neg Q)$
    \item $\vdash (\neg P \Leftrightarrow Q) \Leftrightarrow (P \Leftrightarrow \neg Q)$
    \item $\vdash (P \Leftrightarrow (P \land (P \lor \neg P) ))$
    \item $\vdash (P \Leftrightarrow (P \lor \bot ))$
    \item $\vdash (P \land Q) \Leftrightarrow (Q \land P)$
    \item $\vdash (P \lor  Q) \Leftrightarrow (Q \lor  P)$
    \item $\vdash (\neg P\Rightarrow Q) \Leftrightarrow (\neg Q\Rightarrow P)$
    \item $\{P \land Q\} \vdash (P \lor  Q) \Leftrightarrow (P \Leftrightarrow Q)$
    \item $\{P \lor  Q\} \vdash (P \land Q) \Leftrightarrow (P \Leftrightarrow Q)$
    \item $\vdash (P\Rightarrow Q) \Leftrightarrow (P \Leftrightarrow (P \land Q))$
    \item $\vdash (P\Rightarrow Q) \Leftrightarrow (Q \Leftrightarrow (P \lor  Q))$
    \item $\vdash (P \Leftrightarrow Q) \Leftrightarrow ((P\Rightarrow Q) \land (Q\Rightarrow P))$
    \item $\vdash (P \Leftrightarrow Q) \Leftrightarrow ((P \lor  Q)\Rightarrow (P \land Q))$
    \item $\vdash (P \land Q)  \Leftrightarrow  (P \Leftrightarrow (P\Rightarrow Q))$
    \item $\vdash (P \land Q)  \Leftrightarrow  (Q \Leftrightarrow (Q\Rightarrow P))$
    \item $\vdash (P \lor  Q)  \Leftrightarrow  (P \Leftrightarrow (Q\Rightarrow P))$
    \item $\vdash (P \lor  Q)  \Leftrightarrow  (Q \Leftrightarrow (P\Rightarrow Q))$
    \item $\vdash (P \lor  Q)  \Leftrightarrow  ((P \Leftrightarrow Q)\Rightarrow P)$
    \item $\vdash (P \lor  Q)  \Leftrightarrow  ((P \Leftrightarrow Q)\Rightarrow Q)$
    \item $\vdash (P \Leftrightarrow Q) \Leftrightarrow ((P \land Q) \lor  (\neg P \land \neg Q))$
    \item $\vdash (P \Leftrightarrow Q) \Leftrightarrow ((\neg P \lor  Q) \land (P \lor  \neg Q))$
    \item $\{P \Leftrightarrow Q\} \vdash \neg P \Leftrightarrow \neg Q$
    \item $\{P \Leftrightarrow Q\} \vdash (P \land R) \Leftrightarrow (Q \land R)$
    \item $\{P \Leftrightarrow Q\} \vdash (P \lor  R) \Leftrightarrow (Q \lor  R)$
    \item $\{P \Leftrightarrow Q\} \vdash (P\Rightarrow R) \Leftrightarrow (Q\Rightarrow R)$
    \item $\{P \Leftrightarrow Q)\} \vdash (R\Rightarrow P) \Leftrightarrow (R\Rightarrow Q)$
    \item $\vdash \neg (P \Leftrightarrow \neg P)$
    \item $\vdash ((P \Leftrightarrow Q) \Leftrightarrow Q) \Leftrightarrow P$
    \item $\{P \Leftrightarrow (Q \Leftrightarrow R)\} \vdash (R \Leftrightarrow P) \Leftrightarrow Q$
    \item $\{P \Leftrightarrow (Q \Leftrightarrow R)\} \vdash Q \Leftrightarrow (R \Leftrightarrow P)$
    \item $\{P \Leftrightarrow (Q \Leftrightarrow R)\} \vdash (P \Leftrightarrow Q) \Leftrightarrow R$
    \item $\vdash \neg (P \land Q) \Leftrightarrow \neg P \lor \neg Q$
\end{enumerate}

\begin{exercise}\label{exerc:LPro3}
    Usando o sistema dedutivo demonstre que:
\end{exercise}

\begin{enumerate}
    \item Se $\{\alpha, \beta\} \vdash \gamma$, então $\{\neg(\alpha \Rightarrow \neg \beta)\} \vdash \gamma$
    \item $\{\alpha\} \vdash \neg \neg (\alpha \Rightarrow \neg \beta) \Rightarrow \neg \beta$
    \item Se $\vdash \alpha \Rightarrow \beta$ e $\vdash \alpha \Rightarrow \gamma$, então $\vdash \alpha \Rightarrow \neg (\beta \Rightarrow \neg \gamma)$
    \item Se $\Gamma \vdash \alpha$, então $\Gamma \vdash \neg \alpha \Rightarrow \beta$
    \item Se $\Gamma_1 \vdash \alpha$ e $\Gamma_2 \vDash \alpha \Rightarrow \beta$, então $\Gamma_1 \cup \Gamma_2 \Rightarrow \beta$
    \item Se $\Gamma \vdash \alpha \Rightarrow \beta$ e $\Gamma \vdash \neg \alpha \Rightarrow \gamma$, então $\Gamma \vdash \beta$ ou $\Gamma \vdash \gamma$
    \item Se $\Gamma \vdash \beta$ ou $\Gamma \vdash \gamma$, então $\Gamma \vdash \neg \beta \Rightarrow \gamma$
    \item Se $\Gamma \vdash \alpha$ ou $\Gamma \vdash \beta$, então $\Gamma \vdash \neg (\neg \alpha \land \neg \beta)$
\end{enumerate}

\begin{exercise}\label{exerc:LPro4}
    Considerando o sistema axiomático L demonstre as seguintes relações de consequência.
\end{exercise}

\begin{enumerate}
    \item $\{\alpha\} \vdash_L \beta \Rightarrow (\neg \alpha \Rightarrow \gamma)$
    \item $\vdash_L (\beta \Rightarrow \alpha) \Rightarrow ((\neg \alpha \Rightarrow \beta) \Rightarrow \alpha)$
    \item $\{\alpha, \beta\} \vdash_L \neg (\alpha \Rightarrow \neg \beta)$
    \item $\{\neg (\alpha \Rightarrow \neg \beta)\} \vdash_L \beta$
    \item $\{\neg (\alpha \Rightarrow \neg \beta)\} \vdash_L \alpha$
    \item $\vdash_L (\neg \alpha \Rightarrow \beta) \Rightarrow (\neg \beta \Rightarrow \alpha)$
    \item $\{P, \neg P\} \vdash_L Q$
    \item $\{P \Rightarrow Q, P \Rightarrow \neg Q\} \vdash_L \neg P$
    \item $\{\neg \neg P \Rightarrow Q, \neg \neg P \Rightarrow \neg Q\} \vdash_L \neg P$
    \item $\{\neg P \Rightarrow Q, \neg P \Rightarrow \neg Q\} \vdash_L P$
\end{enumerate}

\begin{exercise}\label{exerc:LPro5}
    Exiba todas as possíveis valorações para as palavras a seguir.
\end{exercise}

\begin{enumerate}
    \item $P \Rightarrow (Q \lor (R \Rightarrow P))$
    \item $\neg (P \Rightarrow \neg (P \Rightarrow Q)))$
    \item $(P \Rightarrow Q) \Rightarrow R) \Rightarrow R$
    \item $P \Rightarrow (Q \Rightarrow P)$
    \item $P \Leftrightarrow (\neg P \Leftrightarrow \neg (P \lor Q))$
    \item $P \Rightarrow ((Q \land \neg R) \Rightarrow S)$
    \item $((P \land Q) \lor (R \land S)) \Rightarrow ((\neg T \land U) \Leftrightarrow (V \lor X))$
    \item $\bot \Rightarrow (\neg (P \lor \neg P) \land S)$
    \item $(X \land \bot) \Rightarrow (P \Leftrightarrow \neg P)$
    \item $\neg(\neg Q \lor \neg \neg T) \Rightarrow (R \Leftrightarrow \neg Z)$
    \item $((P \lor Q) \land \neg P) \Rightarrow Q$
    \item $(P \Rightarrow (Q \Rightarrow R)) \Rightarrow ((P \Rightarrow Q) \Rightarrow (P \Rightarrow R))$
    \item $(P \Rightarrow P) \Rightarrow P$
    \item $((\neg Q) \Rightarrow (\neg P)) \Rightarrow \neg (P \Rightarrow Q)$
    \item $\neg \neg \neg P \Rightarrow \neg (Q \Leftrightarrow \neg \neg R)$
    \item $\neg (P \land Q) \land (Q \land \neg \neg P)$
    \item $(\neg R \land T) \lor ((S \lor \neg X)\land(P \lor (Q \Rightarrow \bot)))$
    \item $(P \land (Q \land R)) \Rightarrow (R \land (P \land Q))$
    \item $(P \Rightarrow Q) \Leftrightarrow (\neg \neg Q \lor \neg P)$
    \item $(P \land Q) \Leftrightarrow \neg (Q \Rightarrow \neg P)$
\end{enumerate}

\begin{exercise}\label{exerc:LPro6}
    Demonstre ou refute as afirmações a seguir.
\end{exercise}

\begin{enumerate}
    \item O conjunto $\{\neg (\neg P \lor \neg Q) \Rightarrow Q, \neg R \lor S, \neg P \}$ tem um único modelo.
    \item O conjunto $\{P \lor (Q \lor R), \neg P \Rightarrow (Q \lor R), Q \Rightarrow \neg(R \lor P)\}$ tem exatamente 3 diferentes modelos.
    \item O conjunto $\{\neg P \land Q, P \Rightarrow Q, \neg Q \lor (\neg P \lor R)\}$ não possui um modelo.
    \item O conjunto $\{Q \Rightarrow \neg R, P \Rightarrow (R \lor Q), S, T\}$ não possui um modelo.
    \item Qualquer valoração é um modelo para o conjunto $\{Q \lor \neg Q, P \Rightarrow P, \neg (\neg R \land R)\}$.
    \item Não existe modelos para o conjunto $\{P \land Q, Q \lor \neg P, Q \Rightarrow \neg S\}$.
    \item O conjunto $\{P \Rightarrow \bot, \neg \neg Q \Rightarrow P\}$ não tem nenhum modelo.
    \item $\{\neg (P \lor Q) \Rightarrow R\} \vDash \neg P \Rightarrow R$
    \item $\{P \lor Q, P \Rightarrow \neg R, \neg Q \Rightarrow S\} \vDash R \lor \neg S$
    \item $\{P \Rightarrow R, Q \Rightarrow R\} \vDash (P \lor Q) \Rightarrow R$
    \item $\{\neg \neg P \Rightarrow Q\} \vDash (R \lor P) \Rightarrow (Q \lor R)$
    \item $\{P \land (Q \lor R)\} \vDash \neg (\neg P \lor \neg Q) \lor (P \land R)$
    \item $\{(P \land Q) \lor (P \land R)\} \vDash P \land (Q \lor R)$
    \item $\{P \lor (Q \land R)\} \vDash (P \lor Q) \land (P \lor R)$
    \item $\{(P \lor Q) \land (P \lor R)\} \vDash \neg \neg P \lor (Q \land R)$
    \item $\{\neg P, \neg \neg Q\} \vDash \neg P \Leftrightarrow Q$
    \item $\{\neg P, \neg Q\} \vDash \neg (\neg P \Leftrightarrow \neg Q)$
    \item $\{\neg P, Q\} \vDash \neg (\neg P \Leftrightarrow Q)$
    \item $\{P \Leftrightarrow \neg Q, \neg P\} \vDash \neg Q$
    \item $\{\neg P \Leftrightarrow Q, \neg Q\} \vDash \neg P$
\end{enumerate}

\begin{exercise}\label{exerc:LPro7}
    Pada cada palavra a seguir encontre a sua família de conjuntos de literais, isto é, o conjunto FCL de cada palavra.
\end{exercise}

\begin{enumerate}
    \item $P \Rightarrow (R \land \neg Q)$
    \item $(P \land Q) \Rightarrow \neg (P \Rightarrow Q)$
    \item $P \Leftrightarrow ((\neg Q \lor \neg R) \Rightarrow S)$
    \item $R \Rightarrow (Q \Leftrightarrow \neg (P \lor Q))$
    \item $(P \Rightarrow Q) \Rightarrow R) \Rightarrow R$
    \item $P \Rightarrow (Q \Rightarrow P)$
    \item $P \Leftrightarrow (\neg P \Leftrightarrow \neg (P \lor Q))$
    \item $P \Rightarrow ((Q \land \neg R) \Rightarrow S)$
    \item $((P \land Q) \lor (R \land S)) \Rightarrow ((\neg T \land U) \Leftrightarrow (V \lor X))$
    \item $(X \land (P \land \neg P)) \Rightarrow (P \Leftrightarrow \neg P)$
    \item $\neg(\neg Q \lor \neg \neg T) \Rightarrow (R \Leftrightarrow \neg Z)$
    \item $((P \lor Q) \land \neg P) \Rightarrow Q$
    \item $(P \Rightarrow (Q \Rightarrow R)) \Rightarrow ((P \Rightarrow Q) \Rightarrow (P \Rightarrow R))$
    \item $((\neg Q) \Rightarrow (\neg P)) \Rightarrow \neg (P \Rightarrow Q)$
    \item $\neg \neg \neg P \Rightarrow \neg (Q \Leftrightarrow \neg \neg R)$
    \item $\neg (P \land Q) \land (Q \land \neg \neg P)$
    \item $(\neg R \land T) \lor ((S \lor \neg X)\land(P \lor (Q \Rightarrow \neg(\neg Q \lor Q))))$
    \item $(P \land (Q \land R)) \Rightarrow (R \land (P \land Q))$
    \item $(P \Rightarrow Q) \Leftrightarrow (\neg \neg Q \lor \neg P)$
    \item $(P \land Q) \Leftrightarrow \neg (Q \Rightarrow \neg P)$
\end{enumerate}

\begin{exercise}
    Prove por refutação as seguintes relações de consequência.
\end{exercise}

\begin{enumerate}
    \item $\{P \Leftrightarrow Q, \neg P\} \vdash_{RES} \neg Q$
    \item $\{P \Leftrightarrow Q, \neg Q\} \vdash_{RES} \neg P$
    \item $\{\neg P \Leftrightarrow Q, P\} \vdash_{RES} \neg Q$
    \item $\{P \Leftrightarrow \neg Q, Q\} \vdash_{RES} \neg P$
    \item $\{\neg (P \Leftrightarrow Q), P\} \vdash_{RES} \neg Q$
    \item $\{\neg (P \Leftrightarrow Q), Q\} \vdash_{RES} \neg P$
    \item $\{\neg (P \Leftrightarrow Q), \neg P\} \vdash_{RES} Q$
    \item $\{\neg (P \Leftrightarrow Q), \neg Q\} \vdash_{RES} P$
    \item $\vdash_{RES} \neg (P \Leftrightarrow Q) \Leftrightarrow \neg (Q \Leftrightarrow P)$
    \item $\vdash_{RES} \neg (P \Leftrightarrow Q) \Leftrightarrow (\neg P \Leftrightarrow Q)$
    \item $\vdash_{RES} \neg (P \Leftrightarrow Q) \Leftrightarrow (P \Leftrightarrow \neg Q)$
    \item $\vdash_{RES} (\neg P \Leftrightarrow Q) \Leftrightarrow (P \Leftrightarrow \neg Q)$
    \item $\vdash_{RES} (P \Leftrightarrow (P \land (P \lor \neg P) ))$
    \item $\vdash_{RES} (P \land Q) \Leftrightarrow (Q \land P)$
    \item $\vdash_{RES} (P \lor  Q) \Leftrightarrow (Q \lor  P)$
    \item $\{\neg P \lor Q\} \vdash_{RES} \neg (P \land \neg Q)$
    \item $\{(P \Rightarrow Q) \Rightarrow R\} \vdash_{RES} (P \land  Q) \Rightarrow R$
    \item $\{P \land  (Q \Rightarrow R)\} \vdash_{RES} (P \Rightarrow Q) \Rightarrow R$
    \item $\{(P \Rightarrow Q) \land  (P \Rightarrow R)\} \vdash_{RES} P \Rightarrow (Q \land  R)$
    \item $\{P \Rightarrow (Q \land  R)\} \vdash_{RES} (P \Rightarrow Q) \land  (P \Rightarrow R)$
    \item $\{P \lor P\} \vdash_{RES} P$
    \item $\{\neg (\neg P \lor \neg Q)\} \vdash_{RES} (P \land Q)$
    \item $\{\neg P \lor \neg Q\} \vdash_{RES} \neg (P \land Q)$
    \item $\{\neg (P \land Q)\} \vdash_{RES} \neg P \lor \neg Q$
    \item $\{\neg (\neg P \land (Q \land \neg R)), Q\} \vdash_{RES} P \lor R$
\end{enumerate}