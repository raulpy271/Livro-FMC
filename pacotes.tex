% ===================================================
% Pacotes MUITO importantes utilizados
% ===================================================

% Usado na construção do índice
\usepackage{makeidx} 
\makeindex

% Pacotes de Matemática
\usepackage{amsmath,amsfonts,amssymb,amsthm}
\usepackage{wasysym}

% ===================================================
% Pacote para e configurações para desenhar
% ===================================================

% Pacote de desenhos
\usepackage{tikz}
\usepackage{tikz-qtree}

\usetikzlibrary{positioning, calc, chains, fit, shapes, automata, trees}

% Pacote para desenhar árvores
\usepackage{forest}

% ===================================================
% Pacote para caixas de texto
% ===================================================
% Pacote usado para inserir caixinha de texto com sombra (shadowbox)
\usepackage{fancybox}

% ===================================================
% Pacote para os diagramas de Fitch
% ===================================================
\usepackage{fitchPS}

% ===================================================
% Pacote para multi-linhas nas tabelas
% ===================================================
\usepackage{multirow}

% ===================================================
% Pacote para introduzir epígrafes nos capítulos 
% ===================================================
\usepackage{epigraph}

% ===================================================
% Pacote para o not em operadres de relação
% ===================================================
\usepackage{centernot}


% ===================================================
% Pacotes e configurações básicas
% ===================================================

% Pacote para trabalhar com imagens
\usepackage{graphicx}
% Defini uma pasta básica para conter imagens
\graphicspath{{Images/}} 

% Pacote de idiomas responsável pelo quebrar e colocar hifém nas palavras
\usepackage[portuguese]{babel}

% Para customizar as listas
\usepackage{enumitem} 
\setlist{nolistsep} 

% Para as tabelas nos padrões do livro 
\usepackage{booktabs}

% Pacote usado para especificar cores
\usepackage{xcolor}

% ===================================================
% Pacotes e configuração de algoritmos
% ===================================================
\usepackage[portuguese,ruled,lined, linesnumbered]{algorithm2e}
\usepackage{algorithmic}

% ===================================================
% Pacote e configuração das fontes
% ===================================================

% Use the Avantgarde font 
\usepackage{avant}

 % Use the Times font
%\usepackage{times}

% Para fontes matemáticas \usepackage{mathptmx} 
%\usepackage{mathptmx} 

% Slightly tweak font spacing for aesthetics
\usepackage{microtype} 

% Para entrada de texto com acentos
\usepackage[utf8]{inputenc}

% Codificação de 8-bits para os caracteres
\usepackage[T1]{fontenc}

% Pacote adicional para a citações
\usepackage{csquotes}

% Pacote para colorir as linhas e colunas das tabelas
\usepackage{colortbl}

% ===================================================
% Pacote e configuração da geometria das páginas
% ===================================================
\usepackage{geometry}

\geometry{
	paper=a4paper, 
	top=3cm,
	bottom=3cm,
	left=3cm,
	right=3cm,
	headheight=14pt,
	footskip=1.4cm,
	headsep=10pt, 
	%showframe, % Uncomment to show how the type block is set on the page
}

% ===================================================
% Definido cores
% ===================================================
\definecolor{ocre}{RGB}{243, 102, 25}
%\definecolor{ocre}{RGB}{143, 102, 150} 
\definecolor{cinzaClaro}{gray}{0.9}
\definecolor{meu-azul}{RGB}{135, 206, 250}
\definecolor{notas-azul}{RGB}{35, 206, 250}
\definecolor{palegreen}{RGB}{152, 251, 152}
\definecolor{ivory}{RGB}{255, 255, 240}
\definecolor{linha-metodo}{RGB}{77, 90, 140}
\definecolor{interno-metodo}{RGB}{210, 230, 228}

% ===================================================
% Pacotes para a bibliografia e o índice
% ===================================================

% Para bibliografia
\usepackage{natbib}

% Usado para melhorar a estética dos títulos
\usepackage{calc} 

% Para construção do índice
\usepackage{makeidx}
\makeindex 


% ===================================================
% Pacote para colocar a plaquinha nos comentários
% ===================================================
\usepackage{manfnt}


% ===================================================
% Pacote para computar e exibir o horário
% ===================================================
\usepackage{fancyhdr}
\usepackage[yyyymmdd,hhmmss]{datetime}